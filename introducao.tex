\chapter*{Introdução}\addcontentsline{toc}{chapter}{Introdução} 

%``O marquês de X tem, como se sabe, um belo gabinete de física, mas a Eletricidade é sua paixão e, se o paganismo ainda vigorasse, ele decerto ergueria altares elétricos. Ele sabia quais são minhas preferências e não ignorava que também sou fã da \emph{Eletromania}. Convidou-me, portanto, para um jantar onde estariam presentes, segundo ele, os medalhões da ordem dos eletrizantes e das eletrizadas''. Conviria conhecer essa eletricidade falada que, sem dúvida, revelaria muito mais sobre a psicologia da época do que sobre sua ciência\footnote{Bachelard, G. \textit{A formação do espírito científico}. p.~41. Ed. Contraponto. Trad. Estela dos Santos de Abreu 1938 -- 1996}.

\citeonline[p.~11--12]{santos_filosofia_2008} elabora a imagem mental \ver{sec:imagem_mental} de uma feira do conhecimento, onde teorias  são antropomorfizadas, escravizadas e vendidas:\ ``determinismo, livre arbítrio,universalismo, relativismo, realismo, construtivismo, marxismo, liberalismo, neoliberalismo, estruturalismo, pós-estruturalismo, modernismo, pós-modernismo, colonialismo, pós-colonialismo, etc.''. As idéias perderam a utilidade para os ex-adeptos, que não estão mais interessados em comprá-las. E vendem aos que supõe algum valor. Para efetuar a venda, é necessário estabelecer uma relação de custo-benefício, definida por meio das respostas às perguntas: ``qual a utilidade que esta ou aquela teoria poderá ter para mim? Qual o preço?''. A valorização ocorre quando esta teoria se torna mais apelativa que aquela: o preço-teto estabelecido é a verdade da resposta. A livre-associação dos vendedores buscará regulamentar as compras e vendas de conhecimentos conforme seu interesse mais fundamental: se todas teorias forem vendidas não existirá teoria para se vender amanhã.

É possível pensar que esta metáfora de um Mercado contemporâneo do conhecimento é uma ficção científica forçada. Mas Santos esclarece que este tema é anterior à formação do espírito científico moderno: no texto satírico \emph{A venda de filosofias} (165), Luciano de Samósata (125 -- 181?),  escreve sobre um mercado estimulado por Zeus e gerenciado por Hermes. Neste mercado é possível comprar escravos como  ``(\ldots) epicurismo, estoicismo, cepticismo peripatético.'':

\begin{citacao}
Hermes atrai os potenciais compradores, todos comerciantes, gritando alto e bom som “À venda! Uma variedade sortida de filosofias vivas! Posições de todo o tipo! Pagamento à vista ou mediante garantia!” (1905: 190). A “mercadoria” vai sendo exposta, os comerciantes vão chegando e têm o direito de interrogar cada uma das filosofias à venda, começando invariavelmente com a pergunta pela utilidade para o comprador e a sua família ou grupo. O preço é estabelecido por Zeus que, por vezes, se limita a aceitar ofertas feitas pelos comerciantes compradores. A venda tem pleno êxito e Hermes termina, ordenando às teorias que deixem de oferecer resistência e sigam com os seus compradores, ao mesmo tempo que avisa o público: “Senhores, esperamos vê-los amanhã. Estaremos oferecendo novos lotes úteis para homens comuns, artistas e comerciantes”   
\end{citacao}

O texto satírico de Luciano de Samósata, descrito por Santos, antecipa em séculos uma característica de um Mercado regulamentado do conhecimento, formado por ``universidades, editoras, resenhas, revistas especializadas, congressos, jornais de divulgação cultural, catálogos, \emph{amazon.com},(\ldots)''\cite[p.~12--13]{santos_filosofia_2008}. A universidade é um caso particular, pois é o setor de validação deste conhecimento.  

Recolhendo esta imagem mental do Mercado de conhecimentos escravizados, sugerimos espelhar. Isto é, uma outra imagem mental, semelhante, é a imagem de uma \metafora{mercado de teorias musicais}{modalismo,tonalismo,\ldots, pop,\ldots, à música de um som, silêncio, \emph{noise}}. Potenciais compradores podem ser compositores, intérpretes, musicólogos. No entanto, no contexto desta pesquisa, a venda foi concretizada primeiro por cientistas da programação e engenheiros, e depois por compositores \cite{mathews_digital_1963,mathews_technology_1969,mathews_groove_1970,roads_interview_1980,park_interview_2009}.


\section*{Conceito de Pensamento Ortopédico}

Alex McLean defende que a metáfora é de importância central para a linguagem, e o faz através da \emph{teoria da Metáfora Conceitual}\apud[p.~32]{lakoff_methaphors_1980}{McLean2011}. Utilizamos a metáfora para exemplificar o pensamento analítico empregado. McLean sugere utilizar texto entre aspas, seguido de outro em caixa alta, como uma estratégia transversal entre o texto de conotação e o texto de denotação.

Santos utiliza três metáforas. A primeira metáfora é o \metafora{pensamento ortopédico}{o mesmo processo especialista utilizado pelo médico responsável em corrigir deformidades do corpo}\disponivelem{http://www.priberam.pt/dlpo/ortopedia}. A segunda é a \metafora{a razão indolente}{insensibilidade com respeito às consequências do processo de correção}: ``A carência a respeito da finitude transforma-se num problema técnico-científico, enquanto a carência a respeito da diversidade infinita é ignorada como um não-problema.'' \cite[p.~15]{santos_filosofia_2008}. O terceiro é \emph{o pensamento abissal} \cite[p.~1--4]{santos_abissal_2007}:

\begin{citacao}
Consiste num sistema de distinções visíveis e invisíveis, sendo que as invisíveis fundamentam as visíveis. As distinções invisíveis são estabelecidas através de linhas radicais que dividem a realidade social em dois universos distintos: o universo  'deste lado da linha' e o universo 'do outro lado da linha'. A divisão é tal que 'o outro lado da linha' desaparece enquanto realidade, torna-se inexistente, e é mesmo produzido como inexistente. (\ldots) \textbf{O pensamento abissal moderno salienta-se pela sua capacidade de produzir e radicalizar distinções.} Contudo, por mais radicais que sejam estas distinções e por mais dramáticas que possam ser as consequências de estar de um ou do outro dos lados destas distinções, elas têm em comum o facto de pertencerem a este lado da linha e de se combinarem para tornar invisível a linha abissal na qual estão fundadas. \end{citacao}

\section*{Hipótese}

A digressão da metáfora do \metafora{conhecimento simbolicamente capitalizado} ilustra, na esfera desta pesquisa, a percepção de uma bricolagem de teorias na improvisação de códigos (\emph{live coding}). Defendemos que o \emph{live coding} é um pensamento ortopédico que possue a particularidade de estimular a bricolagem de conceitos, para elaboração de algoritmos tonais (que não excluem a utilização de materiais sonoros não-tonais). 

\section*{Objetivo}

Investigar um método de análise/criação para uma improvisação de códigos (\emph{live coding}). Será realizado no \autoref{cap:estudos_de_caso} do ponto de vista do \emph{Quadro conceitual de Sistemas Criativos}\ver{sec:csf}. Como exemplo, analisamos uma sonoridade do primeiro algoritmo de \emph{A Study in Keith} (2009) de Andrew Sorensen.

\section*{Estrutura dos Capítulos}

No \autoref{cap:introducao} selecionamos três abordagens na improvisação de códigos, escolhidos por manterem alguma conexão com a improvisação de códigos no contexto musical.  No \autoref{cap:metodologia} apresentamos um modelo de formalização da criatividade, do ponto de vista do Modelo de Improvisação discutido por Alex \citeonline{mclean_music_2006}. No \autoref{cap:estudos_de_caso}, organizamos conceitos de uma sonoridade. Um algoritmo inicial de \emph{A Study in Keith} é carregada de tradição tonal, ao mesmo tempo que é divulgada como uma nova técnica. Por último uma conclusão, mais uma reflexão que procura estabeler uma reflexão do processo de pesquisa do que obtenção de um resultado final. O \autoref{app:A} foi adicionado para expor o material que estimulou o interesse pelo tema discutido.