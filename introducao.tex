\chapter*[Introdução]{Introdução}\addcontentsline{toc}{chapter}{Introdução}

A investigação desta pesquisa iniciou com os trabalhos de Max \citeonline{mathews_digital_1963,mathews_technology_1969,mathews_groove_1970}. Enquanto os dois primeiros apresentam fundamentos do Processamento de Sinais Digitais (DSP), o último representa a viabilidade de performances musicais humanas mediadas pela máquina digital.

De certa forma, tais publicações emergem de um \emph{Programa de Investigação Científica} \cite{lakatos_falsification_1970,neto_lakatos_2008} envolvendo Música e Ciências da Computação. Resultados de pesquisas posteriores, materializaram diversos ambientes de programação musical, entre eles, o CSound\footnote{Disponível em \url{https://csound.github.io/}.}, Max/MSP\footnote{Disponível em \url{http://cycling74.com/products/max}.}, PureData\footnote{Disponível em \url{http://puredata.info/}.} e SuperCollider\footnote{Disponível em \url{https://supercollider.github.io/}.}.

Deste último \emph{software} foi possível extrair um núcleo de pesquisa. Este núcleo é um paradigma de programação conhecido como \emph{livecoding} ou \emph{programação imediata}, e será discutido no \autoref{cap:introducao}:

\begin{citacao}
\emph{Programação imediata} (ou: programação de conversa, programação no fluxo, programação interativa) é um paradigma que inclue a atividade de programação ela mesma como uma operação do programa. Isto significa um programa que não é tomado como ferramenta que cria primeiro, e depois é produtivo, mas um processo de construção dinâmica de descrição e conversação - escrever o código e então se tornar parte da prática musical ou experimental. \cite[Verbete JITLib]{supercollider.org_supercollider_2014}\footnote{Tradução de \emph{Just in time programming (or: conversational programming, live coding , on-the fly-programming, interactive programming) is a paradigm that includes the programming activity itself in the program's operation. This means a program is not taken as a tool that is made first, then to be productive, but a dynamic construction process of description and conversation - writing code thus becoming a closer part of musical or experimental practice.}}
\end{citacao}

É definido por outros autores como improvisação musical eletrônica \cite{collins_origins_2014}, como partitura \cite{magnusson_algorithms_2011}, ou atividade social \cite{mori_analysing_2015,prospero_social_2015}.  

Sugerimos neste trabalho uma definição com base na interpretação do texto de Alex \citeonline{mclean_music_2006}, ``Music improvisation and creative systems''. O \emph{livecoding} como um \emph{Universo de conceitos}, contendo diversos \emph{Espaços conceituais} próprios para diferentes \emph{Modelos de improvisação}. O Modelo de Improvisação de McLean é um desenvolvimento particular com base nos trabalhos de Jeff \citeonline{pressing_improvisation_1987} e \citeonline{wiggins_framework_2006}. 

No \autoref{cap:trabalhos_relacionados}, descrevo alguns espaços conceituais históricos do Universo de posssibilidades do \emph{livecoding}. Mais especificamente, os predecessores como o próprio Max Mathews, Pietro Grossi, \emph{The Hub} e \emph{The League of Automatic Composers}, a emancipação do paradigma de compilação de programas \emph{Just in time}, e os manifestos ``\emph{Live Algorithm Programming and Temporary Organization for its Promotion}'' e ``\emph{Show us your screens}''.

Com multiplicidade do Espaço conceitual da pesquisa, foi necessário reduzir o escopo para um exemplo. Andrew Sorensen busca replicar, durante uma sessão de \emph{livecoding}, publicada na internet (2009) , o estilo de improvisação \emph{jazz} de Keith Jarret. \todo{\tiny está faltando um comentário sobre a conclusão.}