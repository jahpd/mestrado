\chapter*{Introdução}\addcontentsline{toc}{chapter}{Introdução}\label{cap:intro}

Giovanni \citeonline[p.~117]{mori_analysing_2015} define a improvisação de códigos em relação à Música, Imagens em Movimento (animações, vídeos), Dança ou Tecelagem. É importante esclarecer que essa definição denota a polivalência de uma técnica:

\traduzcitacao{\emph{Live coding} é uma técnica artística de improvisação. Pode ser empregada em muitos contextos diferentes de performance: dança, música, imagens em movimento e mesmo tecelagem. Eu concentrei minha atenção no lado musical, que parece ser o mais proeminente.}{Live coding is an improvisatory artistic technique. It can be employed in many different performative contexts: dance, music, moving images and even weaving. I have concentrated my attention on the music side, which seems to be the most prominente}

Assim como Mori, situamos o \emph{live coding} -- \emph{codificação ao vivo} --, ou como sugerimos chamar, \emph{improvisação de códigos}, com um foco musical. Porém uma abordagem demasiadamente especializada não condiz com a polivalência da técnica. Desta forma, buscamos oferecer ao leitor uma visão de paradigmas, maiores e menores, da improvisação com códigos através de:

• Artefatos artísticos produzidos por uma comunidade de artistas-programadores \cite{prospero_social_2015}, com exemplos da tecelagem, dança e música computacional;

• Bases Históricas da prática de improvisar códigos, em um período nomeado como \emph{Live Computer Music};

• Análise de uma proposição que define o algoritmo gerador dos três primeiros blocos de eventos sonoros \cite{mclean_music_2006} de um exemplo que consideramos mantenedor de uma tradição pianística, em um meio permeado da Música Eletrônica para Dançar\footnote{\cfcite{hillegonda_dj_2013}}, ou de um experimentalismo que inclue a \emph{Música-Ruído}.


\section*{Tecelagem}\addcontentsline{toc}{section}{Tecelagem}

Contextualizar a atividade têxtil é uma forma de recontar os primeiros códigos de computador \ver{sec:tecelagem}. Nas palavras do improvisador de códigos Dave \citeonline{griffths_weave2_2015},

 \traduzcitacao{Um dos potenciais da tecelagem que eu fiquei mais interessado é a capacidade de demonstrar fundamentos de \emph{softwares} por fios -- parcialmente tornar a natureza física da computação auto-evidente, mas também como uma maneira de modelar novas formas de aprender e a entender o que são os computadores}{One of the potentials of weaving I’m most interested in is being able to demonstrate fundamentals of software in threads – partly to make the physical nature of computation self evident, but also as a way of designing new ways of learning and understanding what computers are.}
\end{citacao}

Por outro lado, também é uma forma de apresentar artistas-programadores influentes na improvisação de códigos, entre eles Alex McLean, Adrian Ward e Dave Griffths. Os dois primeiros formaram a banda \emph{Slub} nos anos 2000, com a seguinte premissa: utilizar a atividade de programação para realizar uma Música Eletrônica de Dança. Sua primeira reunião foi em 2001, no \emph{Paradiso club} em Amsterdã, durante o festival \emph{Sonic Arts}. Em 2005 o duo participa do festival \emph{Sonar}, sendo que Griffths é convidado a ser membro oficial, o que abre espaço para a inclusão de novas formas práticas, como \emph{games} \cite[p.~138--140]{McLean2011}, e tecelagem.

\section*{Dança}\addcontentsline{toc}{section}{Dança}

A discussão sobre Dança aprensenta a improvisação de códigos de duas formas: a primeira é a Música Eletrônica para Dançar, através do subgênero \emph{algorave}, e a coreografia de um humano controlada por códigos, que nega o som como resultado da improvisação.

O \emph{algorave} não é exlusivo da improvisação de códigos, mas é suportado por uma comunidade de \emph{live coders}. Segundo \citeonline{prospero_social_2015}, esta comunidade cria artefatos artísticos, frutos de relações entre as pessoas, com uma tendência à institucionalização dos artefatos. Uma dessas institucionalizações, que se apoia fortemente no \emph{algorave}, é a organização chamada de TOPLAP \ver{sec:laptoptoplap}, no cenário Europeu.  

O exemplo coreográfico descreve o projeto de Kate Sichio, \emph{Hacking the Body}/\emph{Hacking Choreography}, e coloca em questão o som ou imagem em movimento como resultado da improvisação. Para Sichio, a relação entre a atividade de escrever programas, e a dança, é parte de um trabalho contínuo entre notação de coreografias e a improvisação de movimentos.

%É importante mencionar que, no escopo da pesquisa de Sichio, são raras as referências aos aspectos sonoros. Diferente das indicações estritas de um pentagrama musical, a partitura de dança ofereçe mais um guia para os movimentos do corpo. Isso significa sugerir que, do ponto de vista computacional, a partitura coreográfica é mais próxima do código escrito em uma linguagem de computador do que a partitura musical tradicional. É curioso notar que ocorre como uma assimilação do pensamento algorítmico na Dança, chamado por \citeonline[p.~31]{sichio_hacking_2004}, através de \citeonline[cap.~1, p.~3]{downie_choreography_2005}, de \emph{Sensibilidades Computacionais}. Ou dispositivos metafóricos elaborados por coreógrafos como Merce Cunningham, Trisha Brown, Bill T. Jones, e William Forsythe -- \traducao{mecanismos de generalização e abstração, representação da coreografia e dança como computação}{mechanisms of generalization and abstraction, choreography as representation, dance as computation} \cite[cap.~1, p.~2--4]{downie_choreography_2005}:usica}

\section*{Música Computacional}\addcontentsline{toc}{section}{Música Computacional}

Do ponto de vista da música, apresentamos três tendências da música computacional através de pesquisadores brasileiros. 

O primeiro é o grupo \emph{FooBarBaz}, formado por Gilson Beck, Renato Fabbri, Ricardo Fabbri e Vilson Vieira. Consideramos este grupo como representativo, no Brasil, de regras institucionalizadas por \citeonline[ver \protect\autoref{sec:foobarbaz}, p.~\protect\pageref{sec:foobarbaz}]{ward_live_2004}

O segundo exemplo é a performance \emph{screenBashing} de Magno Caliman brasileiro mais recente de uma \traducao{(\ldots) performance de \emph{livecoding} arquetípica $[$que$]$ envolve programadores escrevendo códigos no palco, com suas telas projetadas para a audiência.}{The archetypal live coding performance involves programmers writing code on stage, with their screens projected for an audience.}\cite[p.~1, ver \protect\autoref{sec:concerto}, p.~\protect\pageref{sec:concerto}]{mclean_tidal_2010}. 

O terceiro apresenta uma abordagem virtual, isto é, uma improvisação de códigos realizada em locais diferentes, com computadores diferentes, através de uma conexão de \emph{internet} \ver{sec:telepresenca}.

%FooBarBaz é um grupo de improvisação de códigos formado por Gilson Beck, Renato Fabbri, Ricardo Fabbri e Vilson Vieira. Sua primeira apresentação foi durante o Festival Contato 2011. Os membros são ativos em um laboratório virtual conhecido como \emph{labMacambira}\disponivelem{http://labmacambira.sourceforge.net/}. Além de cientistas, participam \emph{hackativistas}, ex-programadores do Google, músicos e artistas plásticos interessados em processos criativos com assistência computacional. É interessante notar que as atividades do FooBarBaz estão sincronizadas com algumas das ideologias da improvisação de códigos, com base em regras práticas \ver{sec:showusyourscreens}. Outro ponto interessante deste grupo foi a elaboração de um manifesto próprio, chamado de \emph{Manifesto Freakcoding}, que inclue, como parte executável do manifesto, um ambiente de programação audiovisual chamado \emph{Vivace} \cite{vieira_vivace:_2015}.

%A performance \emph{screenBashing} de Magno Caliman (ver \autoref{fig:screenbashing}) foi realizada durante o XIII ENCUN\footnote{Encontro Nacional de Compositores Universitários em Campinas-SP no ano de 2015.}. A performance consiste no seguinte: Caliman senta-se ao computador, lateralmente à tela de projeção, com uma iluminação de penumbra. O projetor expõe o estado atual de seu \emph{laptop}, que apresenta um editor de texto. O executante começa a programar em linguagem C (ver exemplo abaixo).

%Uma performance virtual de \emph{live coding} é aquela em que dois ou mais executantes, em endereços diferentes de uma rede de computadores. Isso situa três casos, do qual especificaremos um: i) uma rede local, com computadores diferentes, mas com os improvisadores fisicamente próximos; ii) uma rede remota, privada, que comunica um conjunto de pessoas fisicamente distantes; iii) a rede mundial de computadores, onde o navegador se torna o ambiente virtual de criação musical \cite{roberts_web_2013}. Nos três casos, a premissa é compartilhar o mesmo código entre improvisadores-programadores. Podem também compartilhar do mesmo som, mas isso depende de implementações técnicas . 

 \section*{Proto-História}\addcontentsline{toc}{section}{Proto-História}

Em Fóruns de internet, é colocada em discussão origens da prática de improvisar códigos com fins artísticos. Tais origens determinam um período proto-histórico \cite{ward_live_2004}. Com um hiato de duas décadas, artistas-programadores reinvindicam, através de um manifesto, o \emph{status} de criadores do \emph{live coding}.

Um dos consensos para as origens, que está sendo descontruído, é a influência da performance \emph{Water Surfaces} de Ron Kuivila. Uma desconstrução feita por Giovanni Mori sugere que o compositor italiano Pietro Grossi elaborou as primeiras experiências formais, com um paradigma menor da \emph{Computer Music}, contrastante com aquele formado pela divulgação da família MUSIC N de \citeonline{mathews_digital_1963,mathews_technology_1969}. 

Por outro lado, outros paradigmas menores também são formados através da \emph{Live Computer Music} realizada na Baía de São Franscisco, durante o final da década de 1970, e meados da década de 1980, com os grupos \emph{The League of Automatic Composers} e \emph{The Hub}.
 
Após este período proto-histórico, descrevemos um embate acadêmico em torno de um trabalho publicado por \citeonline{schloss_dilemma_2003}, com uma resposta conjunta de McLean, Griffths, Amy Alexander, Adrian Ward, Fredrik Olofsson, Julian Rohrhuber e Nick Collins, \cite{ward_live_2004}. Esta resposta ficou conhecida como o manifesto \emph{Lubeck04}, ou \emph{Show us your screens}, que delimita regras heurísticas para a prática de improvisar códigos.

\section*{A Study in Keith (2009)}

Ao supor ser possível categorizar gostos musicais na improvisação de códigos\footnote{Para mais detalhes destas categorizações em uma rede social, \cfcite{lunhani_alguns_2015}.}, selecionamos um exemplo específico, \emph{A Study in Keith} \cite{sorensen_keith_2009,sorensen_youtube_2014}, para ilustrar o curioso caso do intérprete concertista de \emph{jazz}, readequado para os propósitos do artista-programador.

O registro audiovisual principal segue a seguinte proposição: após a escuta dos Concertos \emph{Sun Bear} de Keith Jarret, Andrew Sorensen improvisou códigos com o ímpeto de automatizar uma improvisação pianística, fato que se consolidou com suas \emph{Disklavier Sessions} (2011).

Embora inspiradas pela atividade perceptiva, o resultado computacional não guarda nenhuma relação harmônica ou melódica com o pianista estadounidense. No entanto, nosso interesse está mais em analisar a proposição que resulta no algoritmo gerador dos três primeiros blocos sonoros de \emph{A Study in Keith}. 