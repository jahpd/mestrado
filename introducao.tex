\chapter*[Introdução]{Introdução}\addcontentsline{toc}{chapter}{Introdução}

A adoção do computador como instrumento musical remontou, neste trabalho, às investigações de Max \citeonline{mathews_digital_1963,mathews_technology_1969,mathews_groove_1970}. Enquanto os dois primeiros apresentam fundamentos do Processamento de Sinais Digitais (DSP) para o fazer musical em \emph{tempo diferido},  o último apresenta a possibilidade de uma performance musical humana mediada pela máquina digital. Através de um sistema de retroalimentação, um ser humano era capaz de sintetizar sons, e manipulá-los em tempo de execução.

De certa forma, tais publicações permitiram a emergência de um \emph{Programa de Investigação Científica} \cite{lakatos_falsification_1970,neto_lakatos_2008} envolvendo Música e Ciências da Computação. Resultados das pesquisas posteriores, foram materializadas na elaboração de diversos ambientes de programação musical, entre eles, o CSound\footnote{Disponível em \url{https://csound.github.io/}.}, Max/MSP\footnote{Disponível em \url{http://cycling74.com/products/max}.}, PureData\footnote{Disponível em \url{http://puredata.info/}.} e SuperCollider\footnote{Disponível em \url{https://supercollider.github.io/}.}.

Deste último \emph{software}, foi possível extrair o objeto de pesquisa. Ao utilizar este ambiente de síntese sonora e composição algorítmica, é possível realizar improvisações musicais, durante a atividade de codificação de um programa. Esta atividade peculiar ficou conhecida como \emph{livecoding}:

\begin{citacao}
\emph{Programação imediata} (ou: programação de conversa, programação no fluxo, programação interativa) é um paradigma que inclue a atividade de programação ela mesma como uma operação do programa. Isto significa um programa que não é tomado como ferramenta que cria primeiro, e depois é produtivo, mas um processo de construção dinâmica de descrição e conversação - escrever o código e então se tornar parte da prática musical ou experimental. \cite[Verbete JITLib]{supercollider.org_supercollider_2014}\footnote{Tradução de \emph{Just in time programming (or: conversational programming, live coding , on-the fly-programming, interactive programming) is a paradigm that includes the programming activity itself in the program's operation. This means a program is not taken as a tool that is made first, then to be productive, but a dynamic construction process of description and conversation - writing code thus becoming a closer part of musical or experimental practice.}}
\end{citacao}

A definição é apropriada para um verbete de um documento técnico. Porém reduz o \emph{livecoding} ao modo de operação com o computador, e não descreve possíveis resultados musicais. Teoricamente, qualquer estética musical pode ser reproduzida durante uma sessão de improvisação (\emph{livecoding session}). Nesse sentido, investiguei o \emph{livecoding} como, nas palavras de Alex \citeonline{mclean_music_2006}, um \emph{universo de conceitos}. Dependendo de como este universo é configurado, diferentes estéticas musicais podem emergir. Este tema será discutido no \autoref{cap:introducao}. 


No \autoref{cap:metodologia} delimito os conceitos deste universo. 

No \autoref{cap:trabalhos_relacionados} descrevo quais são, segundo o Programa de Investigação Científica do \emph{livecoding}, os predecessores da prática. Por outro lado, existem também manifestos que sedimentaram uma ideologia e uma heurística.

No \autoref{cap:estudos_de_caso} investigar multiplicidades estéticas do \emph{live coding} em alguns casos específicos, apresento,, três trabalhos do compositor e programador australiano Andrew Sorensen. O objetivo não é realizar uma análise pormenorizada de casos, mas sim destacar um fenômeno percebido durante a pesquisa. Trabalhei com a hipótese de que, durante os (aproximadamente) 15 anos de emergência do \emph{live coding}, ocorreu uma mistura de várias categorizações musicais no seio da prática. A análise pormenorizada desfoca a atenção na variedade, algo que moveu o pesquisador a pesquisar.