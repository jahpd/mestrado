\chapter*[Introdução]{Introdução}\addcontentsline{toc}{chapter}{Introdução}

A adoção do computador como instrumento musical remonta às investigações de Max \citeonline{mathews_digital_1963,mathews_technology_1969,mathews_groove_1970}.  A dificuldade desta investigação, pelo menos na época, toca a questão de um equilíbrio entre a performance humana, e a resolução dos timbres digitais. De fato, era uma tarefa computacionalmente complicada naquela época. Muitas negociações de capital simbólico (artigos, entrevistas, \emph{softwares}) foram realizadas em mercados do conhecimento de música e de computadores. Estes mercados, círculos universitários ou corporativos, principalmente estadounidenses\footnote{\emph{Bell Labs}, \emph{IBM}, \emph{Princeton}, \emph{MIT}.}, e franceses\footnote{IRCAM}\todo{\tiny Colocar alguma referência concreta sobre alemães e italianos?}, possibilitaram a materialização de programas como CSound, Max/MSP, PureData e SuperCollider, para citar os principais.

Este último \emph{software}, \emph{SuperCollider}, foi o ponta-pé para a escolha de um objeto de estudo. Uma parte da comunidade de músicos que contextualiza este ambiente de programação musical\footnote{Síntese sonora e composição algorítmica. Disponível em \url{https://supercollider.github.io}.}, se apropriou de um paradigma das ciências da computação -- compilação \emph{Just In Time}--, para valorizar outro, o objeto de pesquisa deste trabalho:

\begin{citacao}
\emph{Programação imediata} (ou: programação de conversa, \emph{livecoding} $[$\url{http://www.toplap.org}$]$, programação no fluxo, programação interativa) é um paradigma que inclue a atividade de programação ela mesma como uma operação do programa. Isto significa um programa que não é tomado como ferramenta que cria primeiro, e depois é produtivo, mas um processo de construção dinâmica de descrição e conversação - escrever o código e então se tornar parte da prática musical ou experimental. \cite[Verbete JITLib]{supercollider.org_supercollider_2014}\footnote{Tradução de \emph{Just in time programming (or: conversational programming, live coding , on-the fly-programming, interactive programming) is a paradigm that includes the programming activity itself in the program's operation. This means a program is not taken as a tool that is made first, then to be productive, but a dynamic construction process of description and conversation - writing code thus becoming a closer part of musical or experimental practice.}}
\end{citacao}

A definição é apropriada para um verbete de um documento técnico. Porém reduz  o \emph{livecoding} (ou também \emph{live coding}) a um paradigma técnico, dos códigos programados, ou de um modo técnico de performatizar uma improvisação. Nesse trabalho, investiguei-o como um universo de conceitos. Esse universo de conceitos foi construído por um grupo social urbano, produtor de artefatos técnico-artísticos, sendo que exemplos musicais são numerosos. Embora com desenvolvimento recente, aproximadamente 15 anos, alguns predecessores são lembrados.

Um dos mais precoces, descrito por Giovanni \citeonline{mori_pietro_2015}, é o compositor italiano Pietro Grossi. No final dos anos 70, trata o computador como um caricato do piano. Trabalhando em um laboratório particular, desenvolve programas como TAU, TAU2 e TELETAU. Os dois primeiros abandonam a preocupação timbrística e salientam a questão performática, com apenas uma forma de onda quadrada. O último pode ser descrito como um dos mais antigos programas funcionais para performances musicais remotas.

Segundo Alex \citeonline{mclean_patterns_2009}, no final dos anos 70 e dos anos 80, emergem compositores como John Bischoff, Tim Perkis, Chris Brown, Scot Gresham-Lancaster, Ron Kuivilla e Nicolas Collins. Antes tutoriados por compositores como Alvin Lucier, Darius Milhaud, John Chowning, Robert Ashley e Terry Riley, divulgam a \emph{Live Computer Music} utilizando, principalmente experimentos com micro-controladores e sistemas de \emph{feedback}.

No entanto, para \citeauthoronline{mclean_patterns_2009}, o \emph{live coding} não possui sua identidade cultural até a emergência da organização TOPLAP. Nesse sentido, o manifesto ``\emph{Live Algorithm Programming and Temporary Organization for its Promotion}'' tem um papel fundamental. Este documento identifica não apenas um tipo de performance, mas uma personagem (o \emph{live coder}), um núcleo/heurística para um novo Programa de Investigação Científica (PIC)\footnote{\cfcite{lakatos_falsification_1970,neto_lakatos_2008}.}, e uma rede de manifestações artísticas (delineadas neste trabalho pelos gêneros musicais). A partir disso realizo a seguinte pergunta: \textbf{como é a rede de gêneros músicais quando se fala em \emph{live coding}? ela é simétrica ou assimétrica?}

O \autoref{cap:introducao} discute o \emph{live coding} como \emph{universo de conceitos}. No capítulo \autoref{cap:metodologia}, arituculo este último para almejar um \emph{ecologia de saberes} sobre o \emph{live coding}. Esse método contextualiza, no \autoref{cap:trabalhos_relacionados}, uma discussão sobre uma assimetria de conhecimentos no \emph{live coding}. O \autoref{cap:generos_musicais}, é um prelúdio que contextualiza os resultados no \autoref{cap:resultados}. Os resultados tem por fim confirmar a assimetria de conhecimentos, de maneira quantitativa, em uma rede social (\emph{Soundcloud.com}).