\chapter*{Introdução}\addcontentsline{toc}{chapter}{Introdução} 

Em seu artigo ``A filosofia à venda, a douta ignorância e a aposta de pascal'', \citeonline[p.~11--12]{santos_filosofia_2008} descreve a imagem mental de uma feira do conhecimento, onde teorias  são antropomorfizadas, escravizadas e vendidas. As idéias perderam a utilidade para os ex-adeptos, que não estão mais interessados em comprá-las. E vendem aos que supõem algum valor. Para efetuar a venda, é necessário estabelecer uma relação de custo-benefício, negociadas através de respostas às perguntas: ``qual a utilidade que esta ou aquela teoria poderá ter para mim? Qual o preço?''. A valorização ocorre quando esta teoria se torna mais apelativa que aquela. Com a concorrência, a livre-associação dos vendedores regulamentará compras e vendas de conhecimentos conforme seu interesse mais fundamental: se todas teorias forem vendidas não existirá teoria para se vender amanhã. 

Podemos pensar que Santos realiza uma  metáfora de um Mercado contemporâneo do conhecimento; nos termos deste documento, espaços conceituais da Filosofia e da Economia se conectam por imagens mentais e metáforas. No entanto Santos argumenta que tal imagem mental não é uma metáfora contemporânea, mas é anterior à formação do espírito científico moderno: no texto satírico \emph{A venda de filosofias} (165), Luciano de Samósata (125 -- 181?), escreve sobre um mercado estimulado por Zeus e gerenciado por Hermes:

\begin{citacao}
Hermes atrai os potenciais compradores, todos comerciantes, gritando alto e bom som “À venda! Uma variedade sortida de filosofias vivas! Posições de todo o tipo! Pagamento à vista ou mediante garantia!” (1905: 190). A “mercadoria” vai sendo exposta, os comerciantes vão chegando e têm o direito de interrogar cada uma das filosofias à venda, começando invariavelmente com a pergunta pela utilidade para o comprador e a sua família ou grupo. O preço é estabelecido por Zeus que, por vezes, se limita a aceitar ofertas feitas pelos comerciantes compradores. A venda tem pleno êxito e Hermes termina, ordenando às teorias que deixem de oferecer resistência e sigam com os seus compradores, ao mesmo tempo que avisa o público: “Senhores, esperamos vê-los amanhã. Estaremos oferecendo novos lotes úteis para homens comuns, artistas e comerciantes”   
\end{citacao}

Recolhendo esta imagem mental do Mercado de conhecimentos escravizados, espelhamos a metáfora para o assunto específico deste documento. As filosofias vendidas estão em uma feira chamada \emph{live coding}, que traduzimos por  improvisação de códigos. A tradução literal \emph{codificação ao vivo} é o termo mais adequado; mas não cria a imagem mental necessária para apresentar um modelo lógico de improvisação musical \ver{cap:metodologia}. Segundo \citeonline[p.~129]{McLean2011}, \traducao{O termo \emph{live coding} emergiu em torno de 2003, para descrever a atividade do grupo de praticantes e pesquisadores que começaram a desenvolver novas abordagens na criação de música computacional e animação de vídeo.}{The term \emph{live coding} emerged around 2003, to describe the activity of group of practitioners and researchers who had begun developing new approaches to makin computer music and video animation}. Ali vendem o \emph{jazz}, a música algorítmica, o minimalismo, música ambiental, música \emph{rave}, e a música-ruído. Além disso são vendidas teorias da tecelagem, do audiovisual, da dança, e do lado científico, as ciências antropológicas e cognitivas. Compramos uma amostra, cujo exemplares foram divididos em três grupos (ver Objetivos, p.~\pageref{sec:objetivos}).


\section*{Espaço Conceitual do Pensamento Ortopédico no \emph{live coding}}

Neste documento, atentamos para uma redução do escopo de investigação mais detalhada. No entanto esta própria redução, orientada pela prática científica moderna é comentada e classificada por Santos. São dois os tipos de limitações descritas: o Pensamento Ortopédico e Ecologia de Saberes. Não é nossa intenção discutir a Ecologia de Saberes \footnote{\cfcite{santos_abissal_2007}.}, mas atentar para o Pensamento Ortopédico que permeia esta prática de improvisação (ou pelo menos foi assim que percebemos). 

O conceito de Pensamento Ortopédico pode ele mesmo ser tema de investigação sobre criatividade, limitados em  alguns comentários sobre o termo, para que seja coerente com a metáfora inicial deste trabalho: a descrição simbólica do objeto de pesquisa, como uma feira de filosofias musicais acompanhadas de pesquisas científicas. 

O termo \metafora{pensamento ortopédico}{``o processo especialista do médico responsável em corrigir deformidades do corpo''}\disponivelem{http://www.priberam.pt/dlpo/ortopedia}, pode ser exemplificado, na improvisação de códigos, em dois momentos. No final do primeiro capítulo, \citeonline{ward_live_2004} recorrem aos termos \emph{generativo} e \emph{processo} (como agenciamento do tempo musical). Tais apropriações são revistas por \citeonline{McLean2011}, onde generativo é substituído por descritivo, e o processo é substituído pela bricolagem. 

A segunda metáfora, \metafora{a razão indolente}{``A carência a respeito da finitude transforma-se num problema técnico-científico, enquanto a carência a respeito da diversidade infinita é ignorada como um não-problema''} \cite[p.~15]{santos_filosofia_2008}. A (in)finitude sobre um conceito , a criatividade, transformada em um problema técnico-científico será demonstrada como um método de análise, baseado no \emph{Quadro Conceitual de Sistemas Criativos} \cite{mclean_music_2006,Forth2010}, onde metáforas e imagens mentais de uma improvisação de códigos são reduzidos como um objeto com diversas propriedades. 

A terceira metáfora é \MakeTextUppercase{o pensamento abissal}. \citeonline[p.~1--4]{santos_abissal_2007} discute este conceito da seguinte forma:

\begin{citacao}
Consiste num sistema de distinções visíveis e invisíveis, sendo que as invisíveis fundamentam as visíveis. As distinções invisíveis são estabelecidas através de linhas radicais que dividem a realidade social em dois universos distintos: o universo  'deste lado da linha' e o universo 'do outro lado da linha'. A divisão é tal que 'o outro lado da linha' desaparece enquanto realidade, torna-se inexistente, e é mesmo produzido como inexistente. (\ldots) \textbf{O pensamento abissal moderno salienta-se pela sua capacidade de produzir e radicalizar distinções.} Contudo, por mais radicais que sejam estas distinções e por mais dramáticas que possam ser as consequências de estar de um ou do outro dos lados destas distinções, elas têm em comum o facto de pertencerem a este lado da linha e de se combinarem para tornar invisível a linha abissal na qual estão fundadas.\footnote{Grifo nosso.} 
\end{citacao}

Na improvisação de códigos, as radicalizações das distinções são discutidas a partir de modificações em uma pergunta, e o tratamento dado a elas pela comunidade de improvisadores-programadores europeus. \traducao{Como as pessoas improvisam?}{How do people improvise?}\cite{pressing_improvisation_1987}. No caso específico deste trabalho, a partir de \citeonline{pressing_cognitive_1984}: \traducao{como pode o psicólogo não-especialista acessar os resultados de tais improvisações quando os especialistas podem discordar entre eles?}{how will the non-specialist psychologist assess the results of such improvisations when the specialists may disagree among themselves?}. 

No caso específico deste documento, que investiga a improvisação de códigos, como o musicólogo não-especialista pode acessar os resultados de improvisações de códigos musicais, quando os programadores e musicólogos especialistas podem discordar entre os conceitos e idéias estabelecidas? 

Deste problema radicalizado, outras ordens de conhecimento são acessadas. Nos termos deste trabalho, novos espaços conceituais são combinados, explorados e transformados, o que coloca em xeque a própria especialização da investigação. Neste sentido, buscamos limitar a pesquisa em \emph{i}) Descrever alguns comportamentos criativos psicológicos. \emph{ii}) Descrever comportamentos criativos históricos; \emph{iii}) Descrever um pensamento ortopédico do comportamento criativo; \emph{iv}) Analisar um comportamento criativo que gera um algoritmo de uma sonoridade tonal. 

\section*{Objetivos}\label{sec:objetivos}

• Investigar um Universo de Conceitos sobre a \emph{improvisação de códigos} (\emph{live coding});

• Investigar um Espaço de Conceitos, historicamente restrito, sobre a improvisação de códigos;

• Investigar um método de análise/criação para uma improvisação de códigos;

• Investigar um Espaço Conceitual de uma improvisação de códigos, \emph{A Study in Keith} (2009) de Andrew Sorensen, e seu algoritmo musical em um ciclo de transformação.

\section*{Estrutura dos Capítulos}

Inicialmente, os \autoref{cap:introducao} e \autoref{sec:protohistoria} eram um só capítulo. Com a densidade de informações, foi necessário dividir em dois capítulos. De certa forma, esta divisão é um reflexo do universo psicológico do autor deste documento, e de uma aproximação do universo histórico da improvisação de códigos. No \autoref{cap:introducao} selecionamos três abordagens, escolhidas por manterem alguma conexão com a improvisação de códigos, no contexto musical.  No \autoref{sec:protohistoria} buscamos contextualizar características históricas e musicais da improvisação de códigos. No \autoref{cap:metodologia} apresentamos um modelo de formalização da criatividade, do ponto de vista do Modelo de Improvisação discutido por \citeonline{mclean_music_2006,Forth2010,McLean2011} (segundo objetivo). No \autoref{cap:estudos_de_caso}, organizamos conceitos para analisar o contexto de um algoritmo gerador de uma sonoridade tonal em \emph{A Study in Keith} (2009) de Andrew Sorensen.  O \autoref{app:A} foi adicionado para expor o material que estimulou o interesse pelo tema discutido, onde conectamos a idéia de \emph{nuvem de palavras} ao \emph{Universo de Conceitos}.% O \autoref{app:B} sugere a inclusão de um trabalho de \citeonline{mathews_groove_1970} no âmbito proto-histórico da improvisação de códigos, bem como a consideração de sujeitos sócio-técnicos como a tecnologia JIT.