\chapter*{Introdução}\addcontentsline{toc}{chapter}{Introdução}\label{cap:intro}

No contexto das Músicas produzidas com computadores no final do século vinte e no começo do século vinte e um, discutimos uma técnica que recorre às linguagens textuais de programação para performances artísticas \cite{McLean2011}. Artistas-programadores (e até mesmo técnicos profissionais) chamam-na de \emph{live coding}, de forma que o antropólogo Giovanni \citeonline[p.~117]{mori_analysing_2015} define-a como uma técnica polivalente:

\traduzcitacao{\emph{Live coding} é uma técnica artística de improvisação. Pode ser empregada em muitos contextos diferentes de performance: dança, música, imagens em movimento e mesmo tecelagem. Eu concentrei minha atenção no lado musical, que parece ser o mais notável.}{Live coding is an improvisatory artistic technique. It can be employed in many different performative contexts: dance, music, moving images and even weaving. I have concentrated my attention on the music side, which seems to be the most prominente.}

Mori enfatiza o \emph{live coding} -- \emph{codificação ao vivo} ou \emph{improvisação de códigos} -- como foco de proposições musicais. Durante a pesquisa, fomos desafiados a encarar uma técnica que permite produzir, algumas vezes ao mesmo tempo, Música Eletrônica para Dançar, Música-Erro, Música-Ruído, Jazz, e como será possível observar no primeiro capítulo, performances corporais, visuais e têxteis. Dividir os capítulos deste documento exigiu recortar a definição do pesquisador italiano, com exemplos de Tecelagem e Dança como pertinentes para as definições de base, e Música como discussão histórica e de estudo de caso. Exemplos audiovisuais podem ser tão numerosos quanto os exemplos musicais, de forma que não serão incluídos integralmente, no máximo, como elemento adjacente.

\section*{Capítulo 1}

\subsection*{Tecelagem}

Contextualizamos a improvisação de códigos a partir da tecelagem. Nas palavras do improvisador de códigos Dave \citeonline{griffths_weave2_2015}, a tecelagem possibilita a compreensão dos mecanismos de programação, e dos significados dos códigos de computador:

 \traduzcitacao{Um dos potenciais da tecelagem que eu fiquei mais interessado é a capacidade de demonstrar fundamentos de \emph{softwares} por tecidos -- parcialmente tornar a natureza física da computação auto-evidente, mas também como uma maneira de modelar novas formas de aprender e a entender o que são os computadores}{One of the potentials of weaving I’m most interested in is being able to demonstrate fundamentals of software in threads – partly to make the physical nature of computation self evident, but also as a way of designing new ways of learning and understanding what computers are.}

Por outro lado, descrever uma técnica de improvisação direcionada para a prática de tecelagem é uma forma de costurar Dança e Música. Griffths, ao lado de Alex McLean e Adrian Ward, realizam, no começo dos anos dois mil, aprensentações diversas do \emph{live coding} em terreno inglês com um formato \emph{geek} de Música Eletrônica para Dançar: programação, \emph{VJing}/\emph{DJing} e ambientes \emph{rave} com sonoridade \emph{happy hardcore}. % A primeira reunião do grupo foi realizada em 2001, no \emph{Paradiso club} em Amsterdã, durante o festival \emph{Sonic Arts} com Adrian Ward. Em 2005 o duo participa do festival \emph{Sonar}, sendo que Griffths é convidado a ser membro oficial, o que abre espaço para a inclusão de novas formas práticas, como \emph{games} \cite[p.~138--140]{McLean2011}, e tecelagem.

\subsection*{Dança}

Contrapondo este cenário, apresentamos um trabalho da coreógrafa Kate \citeonline{kate_htb_2015} que merece algum destaque, ao negar o som como resultado da improvisação. Através de improvisações de códigos de uma coreógrafa, os movimentos corporais de outra mulher são controlados nos níveis conceitual e corpóreo.

É importante mencionar que esta coreografia nasce de uma bricolagem de uma partitura de performance, no escopo da pesquisa de Sichio, como uma assimilação de \emph{Sensibilidades Computacionais}, ou dispositivos metafóricos elaborados por coreógrafos como Merce Cunningham, Trisha Brown, Bill T. Jones, e William Forsythe.

%\section*{Capítulo 2:}\addcontentsline{toc}{section}{Música Computacional}

%Do ponto de vista da Música, apresentamos três tendências da música computacional através de improvisadores de códigos brasileiros, fomentados por Universidades Públicas. 

%O primeiro é o grupo \emph{FooBarBaz}, formado por Gilson Beck, Renato Fabbri, Ricardo Fabbri e Vilson Vieira. Consideramos este grupo como representativo, no Brasil, das regras institucionalizadas por \citeonline[ver \protect\autoref{sec:foobarbaz}, p.~\protect\pageref{sec:foobarbaz}]{ward_live_2004}. Sua particularidade toca no fato de serem um grupo híbrido de compositores, físicos e cientistas da computação.

%O segundo exemplo é a performance \emph{screenBashing} de Magno Caliman. Ela ilustra a figura do compositor solista, em um caso particular de uma \traducao{(\ldots) performance de \emph{livecoding} arquetípica $[$que$]$ envolve programadores escrevendo códigos no palco, com suas telas projetadas para a audiência.}{The archetypal live coding performance involves programmers writing code on stage, with their screens projected for an audience.}\cite[p.~1, ver \protect\autoref{sec:concerto}, p.~\protect\pageref{sec:concerto}]{mclean_tidal_2010}. 

%O terceiro apresenta uma abordagem virtual, isto é, uma improvisação de códigos realizada em locais diferentes, com computadores diferentes, através de uma conexão de \emph{internet}. Esta abordagem não é nova, mas ilustra a correlação entre uma universidade pública brasileira e uma instituição estadounidense \ver{sec:telepresenca}.

%FooBarBaz é um grupo de improvisação de códigos formado por Gilson Beck, Renato Fabbri, Ricardo Fabbri e Vilson Vieira. Sua primeira apresentação foi durante o Festival Contato 2011. Os membros são ativos em um laboratório virtual conhecido como \emph{labMacambira}\disponivelem{http://labmacambira.sourceforge.net/}. Além de cientistas, participam \emph{hackativistas}, ex-programadores do Google, músicos e artistas plásticos interessados em processos criativos com assistência computacional. É interessante notar que as atividades do FooBarBaz estão sincronizadas com algumas das ideologias da improvisação de códigos, com base em regras práticas \ver{sec:showusyourscreens}. Outro ponto interessante deste grupo foi a elaboração de um manifesto próprio, chamado de \emph{Manifesto Freakcoding}, que inclue, como parte executável do manifesto, um ambiente de programação audiovisual chamado \emph{Vivace} \cite{vieira_vivace:_2015}.

%A performance \emph{screenBashing} de Magno Caliman (ver \autoref{fig:screenbashing}) foi realizada durante o XIII ENCUN\footnote{Encontro Nacional de Compositores Universitários em Campinas-SP no ano de 2015.}. A performance consiste no seguinte: Caliman senta-se ao computador, lateralmente à tela de projeção, com uma iluminação de penumbra. O projetor expõe o estado atual de seu \emph{laptop}, que apresenta um editor de texto. O executante começa a programar em linguagem C (ver exemplo abaixo).

%Uma performance virtual de \emph{live coding} é aquela em que dois ou mais executantes, em endereços diferentes de uma rede de computadores. Isso situa três casos, do qual especificaremos um: i) uma rede local, com computadores diferentes, mas com os improvisadores fisicamente próximos; ii) uma rede remota, privada, que comunica um conjunto de pessoas fisicamente distantes; iii) a rede mundial de computadores, onde o navegador se torna o ambiente virtual de criação musical \cite{roberts_web_2013}. Nos três casos, a premissa é compartilhar o mesmo código entre improvisadores-programadores. Podem também compartilhar do mesmo som, mas isso depende de implementações técnicas . 

\section*{Capítulo 2}

\subsection*{Proto-História}

Em fóruns de internet, improvisadores de códigos discutem a origem do \emph{live coding} como técnica polivalente. Entre elas, obras audiovisuais de Tom de Fanti, em 1976 \disponivelem{http://lurk.org/groups/livecode/messages/topic/5abPazJSxfegYfVFOzN4T6/}. 

Um consenso da origem na Música destaca a performance \emph{Water Surfaces} do compositor estadounidense Ron Kuivila \cite{ward_live_2004}. Uma desconstrução desta idéia, feita por Giovanni Mori, sugere que o compositor italiano Pietro Grossi elaborou, no começo dos anos 1970, as primeiras experiências formais com um paradigma menor da \emph{Computer Music}, em contraste com aquele paradigma maior formado pela divulgação da família MUSIC N de \citeonline{mathews_digital_1963,mathews_technology_1969}. Outros paradigmas menores também são formados através da \emph{Live Computer Music} da Baía de São Franscisco, durante o final da década de 1970, e meados da década de 1980, com o grupo \emph{The League of Automatic Composers}, embrião de outro, \emph{The Hub}. Este paradigma menor considerava a realização de \emph{shows} (algumas vezes \emph{happenings}) em pequenos estabelecimentos públicos, cuja premissa era a programação de microcontroladores que trocassem informações composicionais. Em outras palavras, a composição colaborativa de compositores automáticos.

\subsection*{Atualidade}

O manifesto \emph{Lubeck04} ou \emph{Show us your screens} é um pequeno texto que estabelece algumas regras de conduta para o artista-programador em situação de performance. Rascunhado em um ônibus por Alex McLean, foi formalizado em um documento maior entitulado \emph{Live algorithm programming and temporary organization for its promotion}, ou simplismente LAPTOP \cite{ward_live_2004}, como uma resposta conjunta de sete artistas-programadores ingleses ao artigo \emph{Using contemporary technology in live performance; the dilemma of the
performer}, publicado por \citeonline{schloss_dilemma_2003}. Estas regras são até hoje utilizadas como condutor técnico para uma improvisação de códigos, seja ela musical, audiovisual, corporal ou têxtil.

\section*{Capítulo 3} 

\subsection*{A Study in Keith (2009)}

Selecionamos \emph{A Study in Keith} \cite{sorensen_keith_2009,sorensen_youtube_2014} como um caso que possibilitou investigar com maior profundidade o momento de elaboração de sua proposição, e sua primeira codificação em texto como um algoritmo executável.

O principal registro audiovisual desta improvisação de códigos aponta uma proposição que será o foco do capítulo: após a escuta dos Concertos \emph{Sun Bear} do pianista e compositor Keith Jarret, Andrew Sorensen improvisa um código com o ímpeto de automatizar uma improvisação pianística, semelhante a um Jazz, com eventos MIDI. Segundo Sorensen, ``\textbf{Não é Keith, mas inspirado por Keith}''.


%Embora inspiradas pela atividade perceptiva, o resultado de \emph{A Study in Keith} não guarda nenhuma relação harmônica ou melódica com o pianista estadounidense. Nosso interesse está mais em analisar a proposição que resulta no algoritmo gerador dos três primeiros blocos sonoros de \emph{A Study in Keith}.