%\begin{agradecimentos}
\newpage
\begin{flushright}
\huge{\textbf{Agradecimentos}}

\small{Ao inominável.
\ \\
À minha família, Jair e Júlia, por todo tipo de apoio, amo vocês! 
\ \\
 Aos amigxs que estão (ou moraram em Juiz de Fora) e que foram fundamentais (de alguma forma) no caminho: Glerm, Anna Flávia, Tiago Rubini, Aline. Aos amigxs de Campinas e São Paulo que de alguma forma me ajudaram ultrapassar a distâncias, nostalgias e amizades: Doshi, Larissa, Dani, Tati, Fábio, Rebechi, Felício).
\ \\
Aos vizinhos que riem apenas de ver a lua, Dhiego e Luisa. Aos colegas de mestrado que também batalharam durante o período vigente, Diego, Nayse, Analu (e Paula). Aos velhos amigos que não vejo a muito
\ \\
Aos Professores Dr. Luiz Eduardo Castelões, Dr. Alexandre Fenerich e Dr. Flávio Luiz Schiavonni por serem fundamentais, no apoio institucional; na sugestão de leituras; na cobrança de prazos; nas críticas; nas conversas sobre Música, Universidade e Tecnologia; ou até mesmo pelo sanduíche de queijo quando a bolsa não caiu. À FAPEMIG por suprir esta lacuna, em um momento delicado nas finanças da Universidade Brasileira.
\ \\
À Professora Rosane Preciosa. Seus brilhos deram novos significados ao \emph{Momentum} de Stockhausen. Ao Professor Dr. Elpp Aravena e Tonil Braz, Engenheiros da Cumbia.

Ao Professor Hans Joachim Koellreutter, para você, redes de silêncios e gestos sonoros. Tocar sua peça foi algo revelador.

Àqueles que passaram pelo caminho, que ajudaram ou atrapalharam. Suas potências permitiram a concepção de um novo universo de possibilidades.}
\end{flushright}

\vfil \ 

\newpage