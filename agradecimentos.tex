%\begin{agradecimentos}
\newpage
\begin{flushright}
\huge{\textbf{Agradecimentos}}

\small{Ao que é impossível pronunciar o verdadeiro Nome, mas cuja potência é o próprio sentido da palavra criativa.
\ \\
Aos sem nome, anônimos da Rua de Juiz de Fora, que deram sentido à fraqueza da pergunta deste trabalho.
\ \\
Para uma família em cada canto do Universo, Jair, Olímpia e Júlia (e humanos pitucha, bila, pituca). 
\ \\
Aos Professores Dr. Luiz Eduardo Castelões, Dr. Alexandre Fenerich e Dr. Flávio Luiz Schiavonni, fundamentais no apoio institucional; na sugestão de leituras; na cobrança de prazos; nas críticas; nas conversas sobre Música. À FAPEMIG por suprir esta lacuna, em um momento delicado nas finanças da Universidade Brasileira.
\ \\
Aos amigxs que estão (ou moraram em Juiz de Fora): Glerm (Soares), Tiago Rubini, Anna Flávia. Aos amigxs de Campinas e São Paulo, que estiveram presentes ou na memória: Celso, Dani, Evandro (bixin), Fábio, Felício, Frederico (d menor), Gustavo, Larissa (humanos zé e diva), Rebechi (humana vivi), Simone (humano foucault), Tati e Ruan (bebê Noah), Dhiego e Luisa , ao pessoal da república Lado C, João, Heron, Igor.  Ao Gustavo e Patrick (pretinha). Ao velho amigo Picchi!
\ \\
\ \\
Ao Professor Hans Joachim Koellreutter pelo centenário.}
\end{flushright}

\vfil \ 

\newpage