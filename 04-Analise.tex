\subsection{Definição do instrumento e do tempo}\label{sec:define_instr}

Seu início é um pequeno comentário que contem o nome do executante e seu email para contato (primeiros sete segundos), bem como a escrita de um código que inicializa o NI-Akoustik (até 0$'$43$''$, ver \autoref{fig:SIK_piano}). 

%%%%%%%%%%%%%%%%%%%%%%%%%%%%%%%%%%%%%%%%%
\begin{example}{Definição de instrumento}
  \centering 
Primeiros eventos musicais gerados a partir das primeiras estruturas válidas de código. \textbf{Fonte}: \cite{sorensen_youtube_2014}.
  \begin{minted}[fontsize=\footnotesize]{cl}
    ;;;;;;;;;;;;;;;;;;;;;;;;;;;;;;;;;;;;;;;;;;;;;;;;;
    ;; Andrew Sorensen andrew@moso.com.au
    (define piano (au:make-node "aumu" "NaDd" "-NI-"))
    (au:connect-node piano 0 *au:output-node* 0)
    (au:update-graph)

    (au:load-preset piano "/tmp/convert_grand.aupreset")
  \end{minted}
  \label{fig:SIK_piano}
\end{example}
%%%%%%%%%%%%%%%%%%%%%%%%%%%%%%%%%%%%%%%%%


Em \tempo{0}{52} Sorensen define um tempo base. Em seguida, Sorensen apaga o código para então iniciar definições de notas (\tempo{0}{54}).

%%%%%%%%%%%%%%%%%%%%%%%%%%%%%%%%%%%%%%%%%
\begin{example}{Definição de tempo}\label{ex:def_tempo}
  \centering
  Definição do tempo base. \textbf{Fonte}: \cite{sorensen_youtube_2014}.
  \begin{minted}[fontsize=\footnotesize]{cl}
    (define *metro* (make-metro 110))
  \end{minted}
  
\end{example}
%%%%%%%%%%%%%%%%%%%%%%%%%%%%%%%%%%%%%%%%%

\subsubsection{Definição de uma sequência de blocos}

Até \tempo{1}{07}, uma rotina auxiliar é definida como um laço iterativo. Porém não encontramos sua especificação no código-fonte do Extempore.

%%%%%%%%%%%%%%%%%%%%%%%%%%%%%%%%%%%%%%%%%
\begin{example}{Definição de uma função auxiliar}
  \begin{minted}[fontsize=\footnotesize]{cl}
    (pc:cb-for-each-p chords piano
                      (pc:make-chord 50 70 2 (pc:diatonic 0 '- degree))
                      dur)
  \end{minted}
\end{example}
%%%%%%%%%%%%%%%%%%%%%%%%%%%%%%%%%%%%%%%%%

Internamente, existe uma rotina que será o cerne de execução de uma nota, acompanhada de uma lista de 4 parâmetros (50, 70, 2):

%%%%%%%%%%%%%%%%%%%%%%%%%%%%%%%%%%%%%%%%%
\begin{example}{Definição de uma nota}\label{fig:SIK_acorde}
  \begin{minted}[fontsize=\footnotesize]{cl}
(pc:make-chord 50 70 2 (pc:diatonic 0 '- degree))
  \end{minted}
\end{example}
%%%%%%%%%%%%%%%%%%%%%%%%%%%%%%%%%%%%%%%%%

A abreviação \verb|pc| significa \emph{pitch class}, e a função \verb|pc:make-chord| significa que a função cria um acorde segundo parâmetros definidos no código-fonte do \emph{Extempore}\disponivelem{https://github.com/digego/extempore/blob/master/libs/core/pc_ivl.xtm}:

\begin{citacao}
\traducao{Cria uma lista do ``número'' $[$com$]$ alturas entre limites ``menor'' e ``maior'' do \emph{pc} dado. Uma divisão dos limites, pelo número de elementos requisitados, decompõem a seleção em extensões diferentes, do qual cada altura é selecionada. \emph{make-chord} tenta selecionar alturas para todos os graus do \emph{pc}. É possível, para  os elementos de um acorde resultarem em -1, se não existir nenhum \emph{pc} para a extensão dada. $[$É$]$ não-determinístico (i.e., resultados variam com o tempo). Argumento 1: limite menor (inclusivo). Argumento 2: Limite maior (exclusivo). Argumento 3: Número de alturas no acorde. Argumento 4: \emph{pitch class} \cite{swift_playingII_2012}.}{Creates a list of "number" pitches between "lower" and "upper" bounds from the given "pc". A division of the bounds by the number of elements requested breaks down the selection into equal ranges from which each pitch is selected.  \emph{make-chord} attempts to select pitches of all degrees of the pc.  It is possible for elements of the returned chord to be -1 if no possible pc is available for the given range. Non-deterministic (i.e. result can vary each time). arg1: lower bound (inclusive).  arg2: upper bound (exclusive). arg3: number of pitches in chord.  arg4: pitch class}
\end{citacao}

Este bloco de códigos cria uma díade, no âmbito de um Ré 2 (MIDI 50) e Si bemol 3 (MIDI 70), dentro de um campo harmônico diatônico (\verb|pc:diatonic|). Por sua vez, este último cria ``um acorde seguindo regras básicas de harmonia diatônca: baseado em uma raiz (0 para C, etc.), maior/menor (\verb|'-| ou \verb|'^|) e graus (i-vii)''\footnote{Tradução nossa de: \emph{(\ldots) a chord following basic diatonic harmony rules: based on root (0 for C etc.) maj/min ('- or '\^) and degree (i-vii)}.}. O resultado não é previsível, e depende de regras específicas de qualidade, que apresentaremos adiante, para classificar os \emph{pitch class} dentro de um grau de um campo harmônico.

\subsubsection {Definição de blocos}\label{sec:define_chords}

Em \tempo{1}{08}, a função \emph{chords} surge no fluxo audiovisual, sem nenhum processo de escrita. Este comportamento caracteriza a utilização de, ou uma cópia/cola de texto, ou de uma execução de um macro do editor de texto usado. Macros são pequenos programas no editor que auxiliam o processo de produção do código. De qualquer forma é importante salientar que o código é preparado \cite{sorensen_youtube_2014}.

%%%%%%%%%%%%%%%%%%%%%%%%%%%%%%%%%%%%%%%%%%%%%%%%
\begin{example}{Algoritmo que define os acordes}

O algoritmo apresenta apenas uma propriedade, tempo (\verb|time|).

\begin{minted}[fontsize=\footnotesize]{cl}
    (define chords
       (lambda (time)
          (for-each (lambda (p)
                       (play-note (*metro* time) piano p 80 (*metro* 'dur dur)))                                 
                    (pc:make-chord 50 70 2 (pc:diatonic 0 (quote -) degree)))
          (callback (*metro* (+ time (* .5 dur))) chords (+ time dur))))

    (chords (*metro* 'get-beat 4.0) 'i 3.0)
\end{minted}
\end{example}
%%%%%%%%%%%%%%%%%%%%%%%%%%%%%%%%%%%%%%%%%%%%%%%%

Primeiro é definida a estratégia transversal, \csf{T}{ask}, com um parâmetro, \verb|time|

%%%%%%%%%%%%%%%%%%%%%%%%%%%%%%%%%%%%%%%%%%%%%%%%
\begin{example}{Estratégia transveral}
\begin{minted}[fontsize=\footnotesize]{cl}
(define chords
   (lambda (time) ... ))
\end{minted}
\end{example}

Seguido de um ``impulso'', ou um estímulo sonoro:

%%%%%%%%%%%%%%%%%%%%%%%%%%%%%%%%%%%%%%%%%%%%%%%%
\begin{example}{Impulso, ou acorde inical}
\begin{minted}[fontsize=\footnotesize]{cl}
     (chords (*metro* 'get-beat 4.0) 'i 3.0)
\end{minted}
\end{example}
%%%%%%%%%%%%%%%%%%%%%%%%%%%%%%%%%%%%%%%%%%%%%%%%

Dentro de \csf{T}{ask}, é executado um laço iterativo, \verb|for-each|, para cada nota de uma díade.

%%%%%%%%%%%%%%%%%%%%%%%%%%%%%%%%%%%%%%%%%%%%%%%%
\begin{example}{Laço iterativo}\label{sec:iterativo}
\begin{minted}[fontsize=\footnotesize]{cl}
(for-each (lambda (p)
             (play-note (*metro* time) piano p 80 (*metro* 'dur dur)))                                 
          (pc:make-chord 50 70 2 (pc:diatonic 0 (quote -) degree)))
\end{minted}
\end{example}
%%%%%%%%%%%%%%%%%%%%%%%%%%%%%%%%%%%%%%%%%%%%%%%%

Cada nota é executada com uma altura \verb|p|, para cada díade definida em \verb|pc:make-chord|, em um momento definido por \verb|time| em relação ao pulso rítmico, com uma duração ainda a ser definida. 

%%%%%%%%%%%%%%%%%%%%%%%%%%%%%%%%%%%%%%%%%%%%%%%%
\begin{example}{Execução da nota}
\begin{minted}[fontsize=\footnotesize]{cl}
(play-note (*metro* time) piano p 80 (*metro* 'dur dur))
\end{minted}
\end{example}
%%%%%%%%%%%%%%%%%%%%%%%%%%%%%%%%%%%%%%%%%%%%%%%%

\verb|play-note| é definido com os seguintes argumentos, momento de execução ($time~\Rightarrow~(*metro* time)$), o instrumento tocado, ($instr~\Rightarrow~piano$), a altura ($pitch~\Rightarrow~p$), o volume ($vol~\Rightarrow~80$) e a duração do acorde ($dur~\Rightarrow~(*metro* 'dur dur)$)\disponivelem{https://github.com/digego/extempore/blob/5aec8b35c6b3058d1c66de7abf752dc667ab61e4/libs/core/instruments-scm.xtm}. 

\subsection{Primeira sonoridade tonal}\label{sec:1aSonoridade}

Este código inicial é então modificado, e finalizado em \tempo{1}{57}, momento em que é possível ouvir uma figura musical (uma classe de objeto \pressingthree{O}{ask}{0}), duas díades, um intervalo de quarta justa entre Sol 2 (MIDI 55) e Dó 3 (MIDI 60). entre Mi bemol 2 (MIDI 51) e Dó 3 (MIDI 60).

%%%%%%%%%%%%%%%%%%%%%%%%%%%%%%%%%%%%%%%%%%%%%%%%
\begin{example}{Estratégia transversal}
\begin{minted}[fontsize=\footnotesize]{cl}
    (define chords
       (lambda (time degree dur)
          (if (member degree '(i)) (set! dur 3.0))
          (for-each (lambda (p)
                       (play-note (*metro* time) piano p
                                  (+ 50 (* 20 (cos (* pi time))))
                                  (*metro* 'dur dur)))
                    (pc:make-chord 50 70 2 (pc:diatonic 0 (quote -) degree)))
          (callback (*metro*) (+ time (* .5 dur))) chords (+ time dur)
                    (random (assoc degree '((i vii)
                                            (vii i))))
                    dur))
    
     (chords (*metro* 'get-beat 4.0) 'i 3.0)
\end{minted}
\end{example}
%%%%%%%%%%%%%%%%%%%%%%%%%%%%%%%%%%%%%%%%%%%%%%%%

Duas transcrições desta primeira figura seguem uma estrutura literal do código, e uma perceptiva. Os primeiros eventos sonoros que ocorrem após o momento de silêncio foram transcritos antes da análise do código. Enquanto Sorensen define um tempo regular de 110 BPM  \ver{ex:def_tempo}, transcrevemos este trecho com um andamento entre 35--40 BPM \ver{fig:ask1}. É interessante notar que tais figuras simbolizam neumas, no caso, um \emph{bipunctum}, ou duas notas repetidas, na mão direita, e na mão esquerda um \emph{clivis}, ou um \traducao{acento agudo com um grave}{Unione dell'accento acuto col grave}\footnote{\cfcite[\emph{idem}]{gasperini_semiografia_1905}.}. No caso específico desta primeira figura, na mão direita, um \emph{bipunctum} , e na mão esquerda, uma \emph{clivis}.

\begin{figure}[!h]
  \centering
  \centering 
  \input{./ask1}
  \input{./ask2}
  \input{./gregorian}
  \caption{Transcrição literal e perceptiva do primeiro evento em \emph{A Study in Keith}. \textbf{Fonte}: autor.}
  \label{fig:ask1}
\end{figure}

 É importante notar que algumas alterações são feitas. A primeira delas é definir outros argumentos para \verb|chords|, como um acorde localizado em um grau de um campo harmônico abstrato, e a duração do acorde executado:

%%%%%%%%%%%%%%%%%%%%%%%%%%%%%%%%%%%%%%%%%%%%%%%%
\begin{example}{Modificação do código original}
\begin{minted}[fontsize=\footnotesize]{cl}
    (define chords
       (lambda (time degree dur) ...))
\end{minted}


A segunda alteração é a indicação de uma situação condicional na primeira transformação da estratégia transversal \csf{T}{ask}. Se o grau a ser executado for uma tônica, no caso, menor, a duração deste acorde será configurada para uma duração de três unidades de tempo -- no caso da nossa transcrição, uma unidade de pulso.

\begin{minted}[fontsize=\footnotesize]{cl}
(define chords
   (lambda (time degree dur)
      (if (member degree '(i)) (set! dur 3.0)) ... ))
\end{minted}

A terceira alteração modifica a intensidade das notas:

\begin{minted}[fontsize=\footnotesize]{cl}
(play-note (*metro* time) piano p
           (+ 50 (* 20 (cos (* pi time))))
           (*metro* 'dur dur))
\end{minted}

Onde a a dinâmica específica ocorre como um comportamento periódico de volumes máximos (fortes), e mínimos (pianos), em, proporcional ao cosseno do tempo instantâneo (\verb|cos (* pi time)|), escalonado para valores MIDI:

\begin{minted}[fontsize=\footnotesize]{cl}
(+ 50 (* 20 (cos (* pi time))))
\end{minted}
\end{example}
%%%%%%%%%%%%%%%%%%%%%%%%%%%%%%%%%%%%%%%%%%%%%%%%

\subsubsection{Regras de qualidade}\label{sec:regras_qualidade}

A estrutura interna da estratégia \verb|chords| explicita algumas regras de qualidade, bem como permite apresentar uma primieira sequência de blocos de eventos \pressingthree{K}{ask}{0}, um conjunto de características \pressingthree{F}{ask}{0} e um pequeno grupo de objetos \pressingthree{O}{ask}{0}. Um conjunto de características é definido pelo momento de execução do evento,\pressingthree{F}{ask}{0}, o grau, \pressingthree{F}{ask}{1}, e a duração deste evento, \pressingthree{F}{ask}{2}. É importante destacar que o momento de execução é relativo ao tempo base, definido dentro do padrão \verb|* metro *| (que será explicado a seguir) de um campo harmônico, onde i representa uma tônica menor, e vii, um acorde de sétimo grau, e a duração deste acorde.

\begin{example}{Regra de qualidade \csf{R}{ask}.}
\begin{minted}[fontsize=\scriptsize]{cl}
( ... (... (callback (*metro*) (+ time (* .5 dur))) chords (+ time dur)
                    (random (assoc degree '((i vii)
                                            (vii i))))
                    dur))
\end{minted}

Cujas características irão gerar blocos de eventos, e sequências de blocos de eventos:

\begin{minted}[fontsize=\scriptsize]{cl}
( ...
  (lambda (time degree dur) ... ))
\end{minted}

O que permite executar como:
\begin{minted}[fontsize=\scriptsize]{cl}
(chords (*metro* 'get-beat 4.0) 'i 3.0)
\end{minted}
\end{example}

\subsubsection{Primeira sequência de blocos de eventos}\label{sec:primeiro_evento}

A \autoref{fig:ask3} indica uma primeira sequência de nêumas, gerados pelo algoritmo acima, em um padrão que é repetido por dois ciclos (blocos de eventos \pressingthree{E}{ask}{0} e \pressingthree{E}{ask}{1}). Durante este tempo, Sorensen realiza uma mudança (1$^o$ ciclo de bricolagem). Esta mudança transita entre o segundo bloco \pressingthree{E}{ask}{1} e terceiro bloco \pressingthree{E}{ask}{2}, e sua exeucção resulta em uma transformação da acentuação, o que termina por colocar, no último compasso deste ciclo, o sétimo grau no tempo forte e o primeiro grau no tempo fraco. 

\subsubsection*{Primeiro Bloco}

O primeiro bloco de eventos \pressingthree{E}{ask}{0} aprensenta um contraponto de primeira espécie, aticulado em tempos fortes e fracos, de acordo com um movimento cadencial $i~\Rightarrow~vii$ \ver{fig:ask3}. Uma característica \pressingthree{F}{ask}{0} do algoritmo é sua direcionalidade em um âmbito de quinta em um número de compassos pares (4 nesse caso). 

\begin{figure}[!h]
  \centering
  \input{./ask3}
  \caption{Primeiros eventos musicais gerados a partir das primeiras estruturas válidas de código. \textbf{Fonte}: autor.}
  \label{fig:ask3}
\end{figure}

A aparente repetição de um mesma classe de eventos sonoros, este mesmo um objeto \pressingthree{O}{ask}{0}, pode ser diferenciada através de figuras neumáticas na mão direita e na mão esquerda \exref{fig:neumaMD1}:

\begin{example}{Transcrição de neumas do primeiro bloco}\label{fig:neumaMD1}

  Notação neumática para a um \emph{bipunctum}, dois \emph{podatus} e uma \emph{clivis} na mão direita. E na mão esquerda, três \emph{clivis} e um bipunctum. 

  \centering{\input{./ask6}}

  Notação schenkeriana que expõe um movimento plagal estrutural na mão direita (\^1-\^4-\^1), com um \emph{porrectus} (um acento agudo, um grave e um agudo) estrutural na mão esquerda (\^5-\^2-\^3).

  \centering{\input{./ask12}}

\end{example}

\subsubsection*{Segundo Bloco}

\begin{figure}[!h]
  \centering
  \input{./ask4}
  \caption{Segundo bloco de eventos musicais. \textbf{Fonte}: autor.}
  \label{fig:ask3}
\end{figure}

Que pode ser reescrito como neumas na mão direita:

\begin{example}{Transcrição de neumas do segundo bloco}\label{fig:neumaMD2}

  Notação neumática para cinco \emph{podatus} e uma \emph{clivis} na mão direita. E na mão esquerda uma \emph{clivis}, um \emph{podatus}, um \emph{bipunctus}, três \emph{podatus}.

  \centering{\input{./ask8}}

 Notação schenkeriana que expõe um movimento autêntica estrutural na mão direita (\^1-\^5-\^1), com uma repetição do \emph{porrectus} anterior (\^5-\^2-\^3).

  \centering{\input{./ask13}}
\end{example}

\subsubsection*{Terceiro Bloco}

Enquanto nos blocos \pressingthree{E}{ask}{0} e \pressingthree{E}{ask}{1} existem eventos significativos do ponto de vista figurativo, o aspecto rítmico é único (um tempo forte no $i$ grau, um tempo fraco na $vii$ grau). É importante destacar que, entre estes blocos, Sorensen realiza uma transformação na estratégia transversal

%%%%%%%%%%%%%%%%%%%%%%%%%%%%%%%%%%%%%%%%%%%%%%%%
\begin{example}{Primeira transformação da estratégia transversal}
\begin{minted}[fontsize=\footnotesize]{cl}
    (define chords
       (lambda (time degree dur)
          (if (member degree '(i)) (set! dur 3.0))
          (for-each (lambda (p)
                       (let* (dur1 (* dur (random '(0.5 1))))
                             (dur2 (- dur dur1)))
                       (play-note (*metro* time) piano p
                                  (+ 50 (* 20 (cos (* pi time))))
                                  (*metro* 'dur dur1))
                       (if (> dur2 0)
                           (play-note (*metro* (+ time dur1)) piano
                                      (pc:relative p (random '(-2 -1 1 2))
                                                   (pc:scale 0 'aeolian))
                                      (+ 50 (* 20 (cos (* pi (+ time dur1)))))
                                      (*metro* 'dur dur2))))
                       (pc:make-chord 50 70 2 (pc:diatonic 0 (quote -) degree)))
          (callback (*metro*) (+ time (* .5 dur)) chords (+ time dur)
                    (random (assoc degree '((i vii)
                                            (v i))))
                    (random (list 1 2 3)))))
    
     (chords (*metro* 'get-beat 4.0) 'i 3.0)
\end{minted}
\end{example}
%%%%%%%%%%%%%%%%%%%%%%%%%%%%%%%%%%%%%%%%%%%%%%%%

O que, durante esta transição, gera uma transformação na acentuação \ver{fig:ask4}.

\begin{figure}{Transcrição do terceiro bloco}
  \centering
  \input{./ask5}
  \caption{Terceiro bloco de eventos musicais. \textbf{Fonte}: autor.}
  \label{fig:ask4}
\end{figure}

\begin{example}{Transcrição de neumas do terceiro bloco}\label{fig:neumaMD3}

  Notação neumática para: \emph{bipunctus}, um \emph{clivis}, um \emph{porrectus}, um \emph{clivis}, um \emph{torculus}, um \emph{clivis}, um \emph{torculus}, dois \emph{clivis} e um \emph{clivis subpunctum} na mão direita. E na mão esquerda um \emph{bipunctus}, um \emph{clivis}, um \emph{podatus}, um \emph{porrectus}, um \emph{torculus}, um \emph{podatus}, um \emph{torculus}, um \emph{climatus}, e um \emph{clivis}.

  \centering{\input{./ask10}}

  A notação schenkeriana sugere um \emph{torculus} estrutural na mão direita (\^1-\^2-\^1), onde existe um tetracorde decendente (\^5-\^4-\^3-\^2). Na mão esquerda, destacamos o ponto mais grave da tessitura aponta para um modalismo (\^5-\^7$b$-\^3), bem como alterações nas estruturas rítmicas do trecho citado.
  

  \centering{\input{./ask14}}
\end{example}

Esta estratégia modifica o o laço iterativo interno de cada altura da díade:

%%%%%%%%%%%%%%%%%%%%%%%%%%%%%%%%%%%%%%%%%%%%%%%%
\begin{example}{Laço iterativo modificado}
\begin{minted}[fontsize=\footnotesize]{cl}
(for-each (lambda (p)
             (let* (dur1 (* dur (random '(0.5 1))))
                   (dur2 (- dur dur1)))
             (play-note (*metro* time) piano p
                        (+ 50 (* 20 (cos (* pi time))))
                        (*metro* 'dur dur1))
             (if (> dur2 0)
                 (play-note (*metro* (+ time dur1)) piano
                            (pc:relative p (random '(-2 -1 1 2))
                                         (pc:scale 0 'aeolian))
                            (+ 50 (* 20 (cos (* pi (+ time dur1)))))
                            (*metro* 'dur dur2))))
             (pc:make-chord 50 70 2 (pc:diatonic 0 (quote -) degree)))
\end{minted}
\end{example}
%%%%%%%%%%%%%%%%%%%%%%%%%%%%%%%%%%%%%%%%%%%%%%%%

A primeira grande mudaça é a definição de duas variáveis internas, através do comando \verb|let| (seja), chamadas \verb|dur1| e\verb|dur2|:

%%%%%%%%%%%%%%%%%%%%%%%%%%%%%%%%%%%%%%%%%%%%%%%%
\begin{example}{Laço iterativo modificado}
\begin{minted}[fontsize=\footnotesize]{cl}
(let* (dur1 (* dur (random '(0.5 1))))
                   (dur2 (- dur dur1)))
\end{minted}
\end{example}
%%%%%%%%%%%%%%%%%%%%%%%%%%%%%%%%%%%%%%%%%%%%%%%%

Estas variáveis irão tornar os ritmos de ambas as mãos independentes. O ritmo da mão direita pode ser mantido ou diminuido (\verb|(* dur (random '(0.5 1))|), enquanto o ritmo da mão esquerda é uma diferença entre uma duração geral, e o ritmo da mão direita. No caso desta nova duração da mão esquerda, é aplicado uma verificação condicional:

%%%%%%%%%%%%%%%%%%%%%%%%%%%%%%%%%%%%%%%%%%%%%%%%
\begin{example}{Laço iterativo modificado}
\begin{minted}[fontsize=\footnotesize]{cl}
(if (> dur2 0)
    (play-note (*metro* (+ time dur1)) piano
               (pc:relative p (random '(-2 -1 1 2))
                            (pc:scale 0 'aeolian))
               (+ 50 (* 20 (cos (* pi (+ time dur1)))))
               (*metro* 'dur dur2)))
\end{minted}
\end{example}
%%%%%%%%%%%%%%%%%%%%%%%%%%%%%%%%%%%%%%%%%%%%%%%%

Se a diferença entre a duração total e a nova duração for inválida (igual a $0$), a nota tocada dependerá do resultado de \verb|pc:relative|. A função \verb|pc:relative| é definida\disponivelem{https://github.com/digego/extempore/blob/master/libs/core/pc_ivl.xtm} como \traducao{seleção de uma altura, de uma classe de alturas relativa à uma dada altura}{select pitch from pitch class relative to a given pitch}. Sua altura serão dadas em passos de segundas menores ou maiores ascendentes/descendentes, relativas ao modo eólico da escala (que no caso transforma a sonoridade tonal em sonoridade modal). 

%%%%%%%%%%%%%%%%%%%%%%%%%%%%%%%%%%%%%%%%%%%%%%%%
\begin{example}{Laço iterativo modificado}
\begin{minted}[fontsize=\footnotesize]{cl}
(pc:relative p (random '(-2 -1 1 2))
             (pc:scale 0 'aeolian))
(+ 50 (* 20 (cos (* pi (+ time dur1)))))
\end{minted}
\end{example}
%%%%%%%%%%%%%%%%%%%%%%%%%%%%%%%%%%%%%%%%%%%%%%%%

O ritmo da mão esquerda será semelhante ao da mão direita. 

%%%%%%%%%%%%%%%%%%%%%%%%%%%%%%%%%%%%%%%%%%%%%%%%
\begin{example}{Laço iterativo modificado}
\begin{minted}[fontsize=\footnotesize]{cl}
(+ 50 (* 20 (cos (* pi (+ time dur1)))))
\end{minted}
\end{example}
%%%%%%%%%%%%%%%%%%%%%%%%%%%%%%%%%%%%%%%%%%%%%%%%

No entanto esta característica é um dos fios condutores de uma seção \pressingthree{K}{ask}{1}, o que excede um objetivo deste documento. Nosso interesse nesta análise foi investigar, através do estudo de contextos, e de diferente notações musicais (código, partitura, neuma e esquema analítico), de uma mesma música, a sonoridade que irá gerar outras sonoridades, no caso desta pesquisa, uma sonoridade com raízes em esquemas tonais.

\section{Discussão}

Este capítulo buscou analisar uma zona de conceitos que permeiam a improvisação \emph{A Study in Keith}, publicada em 2009 por Andrew Sorensen. Levantamos, através do quadro conceitual de sistemas criativos, um conjunto de informações sobre um contexto que estimulou o improvisador-programador para sua realização. A partir de uma outra improivisação,  definida como \emph{referente opcional}, ou os \emph{Concertos Sun Bear}, Sorensen buscou simular um estilo de \emph{jazz} do pianista e compositor Keith Jarret. No entanto, esta relação é metafórica, sendo que destacamos um discurso musical eclesiástico, ou baseado na no quarto grau de uma tonalidade. Na nossa transcrição de uma cadência da abertura do disco Kyoto I (que ela mesma, necessita de verificações), encontramos uma exploração da cadência plagal, através de uma substituição por trítono. Já em Sorensen o procedimento é bastante simplificado, quase pedagógico, de como programar figuras musicais -- interpretadas como neumas --, dentro de uma simples cadência autêntica imperfeita (i -- vii). A partir desta regra, foi possível apontar figuras musicias separadas em três blocos, que formam uma primeira sequência do improviso, separada por uma primeira interrupção de uma próxima sequência, ainda não analisada. Desta forma, foi possível delinear um objetivo da improvisação, que é transformar uma classe de objetos sonoros (cuja caracterísitca é ser um contraponto de primeira espécie, com articulação forte-fraco), dentro de um contorno melódico, em um contraponto de segunda espécie, cuja acentuação é alterada. Por outro lado, nossa análise não contemplou sequências seguintes, o que impediu observar detalhes sobre o processo geral da improvisação. Por fim, descrevemos esta análise como uma experiência preliminar em análise de códigos e, por isso mesmo, o Quadro Conceitual de Sistemas Criativos de Alex McLean, e o Modelo de Improvisação de Jeff Pressing, se apresentam como uma interessante ferramente metodológica. 