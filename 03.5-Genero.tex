Neste capítulo descrevo o \emph{live coding} sob a ótica de categorizações do mercado musical, ou nas palavras de \cite{janotti_jr._a_2003,sa_musica_2006,sa_se_2009}, gêneros musicais. Esta sessão tem por função oferecer ferramentas de conhecimento, para lidar com os dados apresentados no Capítulo 4.

Na \autoref{sec:contextualizacao_genero} realizo uma contextualização do termo \emph{gênero musical}. Na \autoref{sec:contextualizacao_genero_livecoding} contextualizo o gênero musical no \emph{live coding}. 

Ao contextualizar gênero como uma categorização do mercado, é pressuposta dinâmicas sonoras, sociais e corporais dentro um ambiente que gera um capital simbólico. Suponho que parte da produção do \emph{live coding} absorveu  sonoridades de diferentes culturas musicais. Nos lados mais extremos:\begin{inparaenum}[\itshape 1)\upshape]
\item aquilo que será  definido como \emph{música-popular massiva}; e 
\item aquelas tendências conhecidas como \emph{noise} e \emph{glitch}.
\end{inparaenum} Cada uma delas gera um capital simbólo diferente, de acordo com seus contextos sociais (o ambiente de performance, como é performatizada a música, que músicas lembram o que está sendo tocado, a luz utilizada durante a performance, a roupa que se veste na performance, etc.).

Por outro lado, absorver diferentes sonoridades não significa que todas caracterísicas sociais serão incluídas integralmente. Neste sentido, busco contextualizar na \autoref{sec:contextualizacao_genero} o que é gênero musical, como uma articulação entre os sujeitos sociais e mercado regulador. Na \autoref{sec:contextualizacao_genero_livecoding} busco contextualizar o \emph{live coding}...


\section{Categorização de sonoridades no \emph{live coding}}\label{sec:categoriza_som}

O próprio conceito de gênero, segundo Judith Butler, é tão complexo que ``exige um conjunto interdisciplinar e pós-disciplinar de discursos'' (Problemas de gênero, p.12) . Embora essa conceitualização seja própria da crítica à divisão dos gêneros sexuais, ela não deixa de ser elucidar o tratamento dos gêneros musicais no \emph{live coding}.


O gênero musical é um processo de categorização de sonoridades, construtos sociais e modos de performance com o corpo. Isto é, são reconhecidos fatores \emph{intra-musicais} e \emph{extra-musicais} que definem cada gênero musical.

A partir de Simon Frith, Janotti \citeonline[p.~32-37]{janotti_jr._a_2003} discute o gênero musical como ``um modo de definição da música em relação ao mercado, do potencial mercadológico presente na música''. Simone Pereira de \citeonline{sa_musica_2006} reforça esta concepção de uma lógica de mercado. Ao repetir a pergunta de Janotti Jr, podemos ter uma idéia mais clara de gênero musical. \textbf{Com que se parece este som? ou do ponto de vista da indústria,'quem vai consumir esta música?} \apud[p.~15]{janotti_jr._a_2003}{sa_musica_2006}.

Ambos concordam que, o mercado musical é um importante dispositivo para regulação das categorias de gênero. Mas ele sozinho não é capaz de definir como uma sonoridade será classificada . \citeauthoronline{janotti_jr._a_2003} lembra de  ``(\ldots) exemplos que mostram que algumas divisões comerciais, antes de se basearem em gêneros musicais, são efetuadas por padrões temporais, gêneros sexuais e/ou feixes linguísticos e geográficos''. Uma \emph{categorização de sonoridades} existe nas esferas de negociação entre mercado musical e os sujeitos participantes deste mercado, isto é, os ``fãs, músicos, críticos e produtores''. Também existe nas esferas musicais e técnicas, como uma música como é tocada, tipos de vozes, famílias de instrumentos existentes, sons eletronicos, acústicos, ritmo (se existe), melodia (se existe), drones (se existe). Existe não apenas no universo psicoacústico ou estético (fatores \emph{intra-musicais}), mas também nas convenções sociais locais para performatizar este ouvir (ritos), e o capital simbólico derivado. Se de um lado a categorização das sonoridades é um modo de definição da música em relação ao mercado, por outro são ``competências diferenciadas para que se construam determinados quadros de valor em relação a certas expressões musicais''. 

Simone Pereira de \citeonline[p.~4-10]{sa_se_2009} argumenta que \emph{gênero musical} é uma categoria tradicional de classificação no universo da cultura da \emph{música-popular massiva}. Ela mesma não dá conta dos fatores \emph{extra-musicais} em rede. Para os sujeitos (fãs, produtores, críticos) é estar atualizado quanto às novidades de sua área de atuação, locais de eventos, novas técnicas, novos grupos, possivelmente o que se veste e o que se consome em reuniõs sociais (sempre lembrando que este consume envolve capital material). Para o mercado musical, significa saber o que estes sujeitos sociais tocam, ouvem, falam, vestem, dançam, compram. Busco então utilizar o termo  para significar o gênero musical. 

\citeonline{sa_se_2009} ainda define \emph{processos de rotulação}, onde  \emph{disputas simbólicas} permeiam uma \emph{música-popular massiva}. Neste universo, grupos de indivíduos são reduzidos a \emph{comunidades de gosto}. Para o Capítulo 4 busco discutir \begin{inparaenum}[\itshape i)\upshape]:
\item qual é o mercado musical do \emph{live coding}, i.e, quem produz, ouve, grava, vende, compra?
\item é uma música massiva ou se utiliza da música massiva?
\item existe um aspecto \emph{intra-musical} ?
\item quais são os aspectos \emph{extra-musicais}?
\end{inparaenum}

\subsection{Contexto de performance do \emph{live coding}}\label{sec:contextualizacao_genero_livecoding}

No capítulo 4 discuto um estudo de caso a respeito do que se toca e o que se ouve em um mercado específico (\emph{Soundcloud}). Dados são derivados de buscas utilizando \emph{hashtags}. As \emph{hashtags} são um dispositivo utilizado para definir um gosto ou ua prática musicalDeixo ao leitor o julgamento se eles foram adequados ou não para descrever o \emph{live coding}.

Destes resultados observei a rotulação de diversas categorias musicais. Ao que se entende por \emph{live coding}, podemos associar o \emph{algorave}, que por sua está relacionado a categorias como  \emph{electroacoustic}, \emph{noise}, \emph{electronic}, \emph{algorithmic}, \emph{algopop}. Segundo \citeonline{mclean_making_2014}, estas manifestações podem ser contextualizadas em quatro tipos peculiares de performance sociais, com o público ou entre os executantes, com o corpo e uma premissa técnica:

\begin{citacao}
Codificar musica ao vivo traz pressões particulares e oportunidades para o planejamento da linguagem de programação. Para reiterar, isto é onde o programador escreve o código para gerar música, onde um processo em execução contnuamente  assume mudanças no seu código, sem quebrar com o resultado musical. A situação arquetípica tem o programador no palco, com sua tela projetada para a audiência ver seu trabalho. Isso poderia ser tarde da noite com uma audiência de uma casa noturna (\ldots), ou durante o dia para uma audiência sentada  em uma sala de concerto (isto é, perforamnce de um conjunto de \emph{laptops}; Ogborn 2014), ou em uma performance colaborativa de longa duração (isto é, \emph{slow coding}; Hall 007). O executante pode se juntar a outros \emph{live coders} ou  músicos instrumentais, ou mesmo com coreógrafos e dançarinos \cite[p.~63]{mclean_making_2014} \footnote{Tradução nossa de: \emph{Live coding of music brings particular pressures and opportunities to programming language design. To reiterate, this is where a programmer writes code to generate music, where a running process continually takes on changes to its code, without break in the musical output. The archetypal situation has the programmer on stage,with their screen projected so that the audience may see them work.This might be late at night with a dancing nightclub audience (e.g.at an algorave; Collins and McLean 2014), or during the day to aseated concert hall audience (e.g. performance by laptop ensemble;Ogborn 2014), or in more collaborative, long form performance(e.g. slow coding; Hall 2007).The performer may be joined byother live coders or instrument
al musicians, or perhaps even choreographers and dancers (Sicchio 2014), but in any case the programmer will want to enter a state of focu
sed, creative flow and work beyond the pressures at hands.}}
\end{citacao}





%No caso do \emph{live coding}, o paralelo é: o \emph{tudio} está empacotado no \emph{laptop}, e músicos-programadores levam para o palco o próprio processo de produção musical (improvisação através do código). Veremos adiante que uma das características do \emph{live coding} é o fato de não se gravar um disco de áudio, permitindo apenas um registro audiovisual da pergormance. Processos de compra e venda dos produtos culturais não ocorrem em lojas, mas sim em universidades, congressos, festivais e, no caso da prática inglesa e outros países da \emph{Europa}, em casas noturnas. Neste ponto o \emph{live coding} toca o aspecto de música massiva; existe uma palavra singular para esta configuração de uma música cerebral, programada , e algoritmizada em espaços de entretenimento chamada \emph{algorave}.

%Ocorre aqui uma simbiose entre as técnicas da Música Algorítmica, \emph{Computer Music} e espaços de entretenimento

%Tais regimes de audição no \emph{live coding} podem ser muitos, dependentes de um \emph{universo de conceitos} \cite{mclean_music_2006}; para poder conectar conceitos, diagramas, imagens e leituras, adotei o método de pesquisa proposto  por \citeonline{santos_filosofia_2008}, buscando observar na \autoref{conjunto_conhecimentos}, o \emph{universo de conceitos do live coding} como um \emph{conjunto limitado de conhecimentos}.

%Durante a contextualização desse conceito, ocorrerá o processo de desconstrução do diagrama de conjunto de cohecimentos, na \autoref{mapeamento_relacoes}. Na seção \autoref{livecoding_eh_musica} buscarei fazer um levantamento de conhecimentos composicionais (estéticas e/ou técnicas) que estão incluídos nos discursos realizados pelos autores que praticam o \emph{live coding}.

%Imaginando aqui conjuntos gerais A, B, C, \ldots, existe um universo U de conceitos. Essa nomenclatura

%\subsection{Esclarecimento da notação de vínculos}

%As sessões supracitadas podem começar com uma proposição do tipo: $A~\Leftrightarrow~B~\Leftrightarrow~C$. A, B e C são um conjunto de conhecimentos, no caso deste trabalho, dependente do termo \emph{live coding}. O símbolo $\Leftrightarrow$ foi adotado para significar que uma área de conhecimento A está em relação com outra área de conhecimento B; esta forma também indica que mais termos podem ser relacionados: A $\Leftrightarrow$ B $\Leftrightarrow$ C descreve uma relação direta de A com B e indireta entre A e C.