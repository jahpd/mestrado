% resumo em português
\setlength{\absparsep}{18pt} % ajusta o espaçamento dos parágrafos do resumo
\begin{resumo}
Esta pesquisa foi construída a partir da seguinte pergunta: ao discutir uma prática conhecida como \textit{live coding}, quais categorizações sonoras (ou gêneros musicais) são contextualizadas? No decorrer da pesquisa encontramos os mais variados exemplos, desde o \emph{noise}, música eletrônica de pista e \emph{jazz}.

A variedade dos diferentes exemplos da pergunta forçou a redução do \emph{universo de conceitos} inerente às categorizações sonoras estudadas. Neste sentido, o texto deste trabalho busca partir de um suposto espaço conceitual generalizado (\autoref{cap:introducao}); um método assimétrico deste espaço conceitual, mas que lembre das multipplicidades (\autoref{cap:metodologia}); uma contextualização histórica, anterior à concepção deste espaço conceitual (\autoref{cap:trabalhos_relacionados}); apresentação de um caso do compositor Australiano Andrew Sorensen (\autoref{cap:estudos_de_caso}).

Este caso foi escolhido por representar a replicação, em uma improvisação com o computador, de um estilo de improvisação de um pianista. Mais especificamente, o \emph{live coding} de Sorensen resulta em uma tentativa de replicar o estilo de \emph{Keith Jarret}, em um Piano MIDI, dos concertos \emph{Sun Bear}, gravados em 1976 e lançados em 1979 pela ECM.

\textbf{Palavras-chaves}: \textit{Live coding}. Categorizações Sonoras. Sorensen
\end{resumo}

%%%%%%%%%% traduçoes resumo
\begin{comment}
% resumo em inglês
\begin{resumo}[Abstract]
 \begin{otherlanguage*}{english}
   This is the english abstract.

   \vspace{\onelineskip}
 
   \noindent 
   \textbf{Key-words}: latex. abntex. text editoration.
 \end{otherlanguage*}
\end{resumo}

% resumo em francês 
\begin{resumo}[Résumé]
 \begi'n{otherlanguage*}{french}
    Il s'agit d'un résumé en français.
 
   \textbf{Mots-clés}: latex. abntex. publication de textes.
 \end{otherlanguage*}
\end{resumo}

% resumo em espanhol
\begin{resumo}[Resumen]
 \begin{otherlanguage*}{spanish}
   Este es el resumen en español.
  
   \textbf{Palabras clave}: latex. abntex. publicación de textos.
 \end{otherlanguage*}
\end{resumo}
% ---
\end{comment}