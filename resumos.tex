% resumo em português
\setlength{\absparsep}{18pt} % ajusta o espaçamento dos parágrafos do resumo
\begin{resumo}
Esta pesquisa foi construída a partir da seguinte pergunta: ao discutir uma prática conhecida como \textit{live coding}, quais categorizações sonoras (ou gêneros musicais) são contextualizadas? No decorrer da pesquisa encontramos os mais variados exemplos, desde o \emph{noise}, música eletrônica de pista e \emph{jazz}.

A variedade dos diferentes exemplos da pergunta forçou a redução de, nas palavras de \citeonline{wiggins_framework_2006} e Alex \citeonline{mclean_music_2006}, nosso \emph{espaço conceitual}, inerente às categorizações sonoras pesquisadas (\autoref{cap:introducao}).

Um método assimétrico, inspirado no sociólogo Boaventura de Souza \citeonline{santos_filosofia_2008,santos_abissal_2007} (e nos espaços conceituais do parágrafo anterior), propõe um tratamento histórico de conceitos do \emph{livecoding}, para poder lidar com os conceitos específicos de um exemplo (\autoref{cap:metodologia})

O tratamento histórico será apresentado no \autoref{cap:trabalhos_relacionados}.

O Estudo de caso específico, a sessão de \emph{livecoding} \emph{Study in Keith}, do compositor e pesquisador Australiano Andrew \citeonline{sorensen_keith_2009} será apresentado no \autoref{cap:estudos_de_caso}. Seguiremos com ferramentas analíticas da peça baseada nos espaços conceituais apresentados no \autoref{cap:introducao}. Este caso foi escolhido por caracterizar uma \emph{replicação do estilo de improvisação} do compositor e pianista Keith Jarret, durante os concertos \emph{Sun Bear}, gravados em 1976 e lançados em 1979 pela ECM.


\vspace{\onelineskip}
\noindent
\textbf{Palavras-chaves}: \textit{Live coding}. \emph{Study in Keith}. Sistemas criativos. Jazz.
\end{resumo}

%%%%%%%%%% traduçoes resumo
% resumo em inglês
\begin{comment}
\begin{resumo}[Abstract]
 \begin{otherlanguage*}{english}
This research was built on the following question: when discussing a practice known as live coding, which sound categorizations (or genres) are contextualized? During the search we will find many examples, from the noise, music
track and jazz electronics.

The variety of different examples of the question forced the reduction in the words of
Wiggins (2006) and Alex McLean (2006), our conceptual space, inherent in categorizations
sound surveyed (Chapter 1).

An asymmetrical method, inspired by the sociologist Boaventura de Souza Santos (2008), Santos (2007) (and conceptual spaces of the previous paragraph), proposes a historical treatment livecoding of the concepts in order to deal with the specific concepts of an example (Chapter 2).

The historical treatment will be presented in Chapter 3.

The study of specific case, the livecoding session \emph{Study in Keith}, from the australian composer and researcher Andrew Sorensen (2012), will be presented in Chapter 4. We will follow with analytical tools based on conceptual piece spaces presented in Chapter 1. This case was chosen to characterize a replication of the improvisational style of the composer and pianist Keith Jarret, during the Sun Bear concerts, recorded in 1976 and published in 1979 by ECM.
   \vspace{\onelineskip}
 
   \noindent 
   \textbf{Key-words}: latex. abntex. text editoration.
 \end{otherlanguage*}
\end{resumo}

% resumo em francês 
\begin{comment}

\begin{resumo}[Résumé]
 \begi'n{otherlanguage*}{french}
    Il s'agit d'un résumé en français.
 
   \textbf{Mots-clés}: latex. abntex. publication de textes.
 \end{otherlanguage*}
\end{resumo}

% resumo em espanhol
\begin{resumo}[Resumen]
 \begin{otherlanguage*}{spanish}
   Este es el resumen en español.
  
   \textbf{Palabras clave}: latex. abntex. publicación de textos.
 \end{otherlanguage*}
\end{resumo}
% ---
\end{comment}