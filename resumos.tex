% resumo em português
\setlength{\absparsep}{18pt} % ajusta o espaçamento dos parágrafos do resumo
\begin{resumo}
Este trabalho discute o \emph{live coding},  uma técnica de improvisação audiovisual, como ferramenta para aglutinação de conceitos. O tema é derivado do Modelo de Improvisação de Alex \citeonline{mclean_music_2006}, um dos precurssores do \emph{live coding} em sua forma corrente.

A formalização deste Universo é acompanhada de uma reflexão sobre a multiplicidade de Espaços Conceituais possíveis, seus limites e transições. Um primeiro espaço conceitual é investigado como uma construção histórica das regras heurísticas do \emph{livecoding}. O segundo espaço conceitual é uma algutinação; um estilo de improvisação de \emph{jazz} é apropriado para elaboração da improvisação \emph{Study in Keith} de Andrew \citeonline{sorensen_keith_2009}.

Mais do que a descrição formal para um modelo de improvisação, para esta performance específica, a intenção deste trabalho é construir espaços conceituais para trabalhos posteriores, bem como oferecer uma pequena introdução em língua portuguesa sobre o que é \emph{live coding}. 

\vspace{\onelineskip}
\noindent
\textbf{Palavras-chaves}: \textit{Livecoding}. \emph{Study in Keith}. Sistemas criativos. Andrew Sorensen. Keith Jarret.
\end{resumo}

%%%%%%%%%% traduçoes resumo
% resumo em inglês
\begin{comment}
\begin{resumo}[Abstract]
 \begin{otherlanguage*}{english}

   \vspace{\onelineskip}
 
   \noindent 
   \textbf{Key-words}: latex. abntex. text editoration.
 \end{otherlanguage*}
\end{resumo}

% resumo em francês 
\begin{comment}

\begin{resumo}[Résumé]
 \begi'n{otherlanguage*}{french}
    Il s'agit d'un résumé en français.
 
   \textbf{Mots-clés}: latex. abntex. publication de textes.
 \end{otherlanguage*}
\end{resumo}

% resumo em espanhol
\begin{resumo}[Resumen]
 \begin{otherlanguage*}{spanish}
   Este es el resumen en español.
  
   \textbf{Palabras clave}: latex. abntex. publicación de textos.
 \end{otherlanguage*}
\end{resumo}
% ---
\end{comment}