
% resumo em português
\setlength{\absparsep}{18pt} % ajusta o espaçamento dos parágrafos do resumo
\begin{resumo}

O recorte de estudo deste documento é uma técnica de improvisação com computadores nomeada como \emph{live coding}. Para definir um termo pouco explorado em língua portuguesa, consideramos expor a pluralidade desta técnica, que sugerimos traduzir como \emph{improvisação de códigos}.

No \autoref{cap:introducao} (ver p.~\pageref{cap:introducao}), apresentamos artefatos (códigos improvisados) e seus criadores (\emph{live coders}). Estes artefatos possuem fronteiras conceituais, observadas do ponto de vista de uma proposição na Introdução (ver p.~\pageref{cap:intro}) que considera a Música como meio predominante desta técnica.

Esta proposição exigiu organizar uma descrição que justifique a Música como recurso histórico \ver{cap:protohistoria}. Ali, apresentamos um período proto-histórico (\emph{live computer music}), e um manifesto \cite{ward_live_2004}, que fixa o termo \emph{live coding}.

No \autoref{cap:estudos_de_caso} (ver p.~\pageref{cap:estudos_de_caso}) analisamos uma proposição de \emph{A Study in Keith} de \citeonline{sorensen_keith_2009}. Mais especificamente a oração ``Não é bem Keith, mas inspirado por Keith'', pela comparação de um fragmento de seu referente (Concertos \emph{Sun Bears} de Keith Jarret) com os três primeiros blocos de eventos sonoros gerados pelo código em linguagem LISP.

Duas são as contribuições deste trabalho para a musicologia: \emph{i}) um documento descritivo de uma história ainda pouco organizada pela comunidade de artistas-programadores; e \emph{ii}) formalização de um método de análise de proposições em uma improvisação de códigos.  

\vspace{\onelineskip}
\noindent
\textbf{Palavras-chaves}: Improvisação de códigos. Música Computacional, \emph{A Study in Keith}.
\end{resumo}

%%%%%%%%%% traduçoes resumo
% resumo em inglês
\begin{comment}
\begin{resumo}[Abstract]
 \begin{otherlanguage*}{english}

   \vspace{\onelineskip}
 
   \noindent 
   \textbf{Key-words}: latex. abntex. text editoration.
 \end{otherlanguage*}
\end{resumo}

% resumo em francês 
\begin{comment}

\begin{resumo}[Résumé]
 \begi'n{otherlanguage*}{french}
    Il s'agit d'un résumé en français.
 
   \textbf{Mots-clés}: latex. abntex. publication de textes.
 \end{otherlanguage*}
\end{resumo}

% resumo em espanhol
\begin{resumo}[Resumen]
 \begin{otherlanguage*}{spanish}
   Este es el resumen en español.
  
   \textbf{Palabras clave}: latex. abntex. publicación de textos.
 \end{otherlanguage*}
\end{resumo}
% ---
\end{comment}