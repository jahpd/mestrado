% resumo em português
\setlength{\absparsep}{18pt} % ajusta o espaçamento dos parágrafos do resumo
\begin{resumo}
O \emph{cantus firmus} desta pesquisa foi construído a partir de uma pergunta. Ao discutir uma prática conhecida como \textit{live coding}, quais categorizações sonoras (ou gêneros musicais) são contextualizadas? 

Contudo, dada a variedade dos diferentes exemplos da pergunta,  foi necessário reduzir o \emph{universo de conceitos} inerente às categorizações sonoras estudadas. Neste sentido, o texto deste trabalho busca partir de um suposto espaço conceitual generalizado (\autoref{cap:introducao}); um método de pesquisa que contemple as multiplicidades deste espaço conceitual (\autoref{cap:metodologia}); uma contextualização histórica, anterior à concepção deste espaço conceitual (\autoref{cap:trabalhos_relacionados}); apresentação de alguns casos do compositor Australiano Andrew Sorensen (\autoref{cap:estudos_de_caso}).

Estes casos foram escolhidos por representarem, em um mesma pessoa, uma plaralidade de práticas musicais. Mais especificamente, com uma mesma técnica de improvisação (ou \emph{live coding}), Sorensen improvisa \emph{Jazz}, \emph{Minimalismo} e \emph{Música eletrônica de Pista}.

\textbf{Palavras-chaves}: \textit{Live coding}. Categorizações Sonoras. Sorensen
\end{resumo}

%%%%%%%%%% traduçoes resumo
\begin{comment}
% resumo em inglês
\begin{resumo}[Abstract]
 \begin{otherlanguage*}{english}
   This is the english abstract.

   \vspace{\onelineskip}
 
   \noindent 
   \textbf{Key-words}: latex. abntex. text editoration.
 \end{otherlanguage*}
\end{resumo}

% resumo em francês 
\begin{resumo}[Résumé]
 \begi'n{otherlanguage*}{french}
    Il s'agit d'un résumé en français.
 
   \textbf{Mots-clés}: latex. abntex. publication de textes.
 \end{otherlanguage*}
\end{resumo}

% resumo em espanhol
\begin{resumo}[Resumen]
 \begin{otherlanguage*}{spanish}
   Este es el resumen en español.
  
   \textbf{Palabras clave}: latex. abntex. publicación de textos.
 \end{otherlanguage*}
\end{resumo}
% ---
\end{comment}