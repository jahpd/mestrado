% resumo em português
\setlength{\absparsep}{18pt} % ajusta o espaçamento dos parágrafos do resumo
\begin{resumo}
Esta pesquisa iniciou com a seguinte pergunta: ao discutir uma técnica de improvisação conhecida como \textit{livecoding}, quais categorizações sonoras podemos contextualizar? Durante a pesquisa, encontramos exemplos variados; \emph{noise}, música eletrônica de pista e \emph{jazz}. O objeto de pesquisa se transformou, o método foi modificado, e uma nova pergunta surgiu.

No \autoref{cap:introducao}, escolhemos conceitos do que é \emph{livecoding}. Reduzimos, nas palavras de \citeonline{wiggins_framework_2006}, o \emph{Espaço conceitual} da pesquisa. Este termo surge de pesquisas sobre criatividade de Margaret Boden. Por outro lado, o tema é derivado para os Modelos de Improvisação de Jeff \citeonline{pressing_improvisation_1987} e revisitado, com viés wigginsiano, por Alex \citeonline{mclean_music_2006}.

No \autoref{cap:metodologia} procuramos descrever formalmente o Espaço conceitual da pesquisa. Ela correlaciona dois outros Espaços conceituais. O primeiro é o próprio \emph{livecoding}. O segundo é um exemplo específico de \emph{livecoding}, \emph{Study in Keith}, do compositor e pesquisador Australiano Andrew \citeonline{sorensen_keith_2009}. No entanto, a natureza de um Espaço conceitual é multidimensional. Nas palavras de Boaventura de Souza \citeonline{santos_abissal_200,santos_filosofia_2008}, cada instância possui sua própria uma \emph{Ecologia de saberes}.

Uma ecologia que contextualize o primeiro Espaço conceitual, é proposta no \autoref{cap:trabalhos_relacionados}, como um tratamento histórico dos comportamentos criativos relacionados ao \emph{livecoding}.

Uma ecologia que contextualize o segundo Espaço conceitual, é proposta no \autoref{cap:estudos_de_caso}, e discute a influência dos Concertos \emph{Sun Bear} (1976-1979) de Keith Jarret em \emph{Study in Keith} no comportamento criativo de Sorensen.

Neste sentido, nosso trabalho visa mais analisar o comportamento criativo do caso estudado. 

\vspace{\onelineskip}
\noindent
\textbf{Palavras-chaves}: \textit{Livecoding}. \emph{Study in Keith}. Sistemas criativos. Andrew Sorensen. Keith Jarret.
\end{resumo}

%%%%%%%%%% traduçoes resumo
% resumo em inglês
\begin{comment}
\begin{resumo}[Abstract]
 \begin{otherlanguage*}{english}
This research was built on the following question: when discussing a practice known as live coding, which sound categorizations (or genres) are contextualized? During the search we will find many examples, from the noise, music
track and jazz electronics.

The variety of different examples of the question forced the reduction in the words of
Wiggins (2006) and Alex McLean (2006), our conceptual space, inherent in categorizations
sound surveyed (Chapter 1).

An asymmetrical method, inspired by the sociologist Boaventura de Souza Santos (2008), Santos (2007) (and conceptual spaces of the previous paragraph), proposes a historical treatment livecoding of the concepts in order to deal with the specific concepts of an example (Chapter 2).

The historical treatment will be presented in Chapter 3.

The study of specific case, the livecoding session \emph{Study in Keith}, from the australian composer and researcher Andrew Sorensen (2012), will be presented in Chapter 4. We will follow with analytical tools based on conceptual piece spaces presented in Chapter 1. This case was chosen to characterize a replication of the improvisational style of the composer and pianist Keith Jarret, during the Sun Bear concerts, recorded in 1976 and published in 1979 by ECM.
   \vspace{\onelineskip}
 
   \noindent 
   \textbf{Key-words}: latex. abntex. text editoration.
 \end{otherlanguage*}
\end{resumo}

% resumo em francês 
\begin{comment}

\begin{resumo}[Résumé]
 \begi'n{otherlanguage*}{french}
    Il s'agit d'un résumé en français.
 
   \textbf{Mots-clés}: latex. abntex. publication de textes.
 \end{otherlanguage*}
\end{resumo}

% resumo em espanhol
\begin{resumo}[Resumen]
 \begin{otherlanguage*}{spanish}
   Este es el resumen en español.
  
   \textbf{Palabras clave}: latex. abntex. publicación de textos.
 \end{otherlanguage*}
\end{resumo}
% ---
\end{comment}