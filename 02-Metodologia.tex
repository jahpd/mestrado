\citeonline{mclean_visualisation_2010} corrobora com definição de \emph{live coding} de \citeonline{blackwell_programming_2005}. Enquanto primeiro afirma que o \emph{live coding} é uma  ``improvisação de vídeo e/ou música usando \textbf{linguagens de computador} que tem se desenvolvido em um campo ativo de pequisa e prática artística ao longo da última década'' \footnote{Tradução parcial nossa de: \emph{Live coding , the improvisation of video and/or music using computer language, has developed into an active field of research and arts pratice over the last decade}. Grifo nosso}, o segundo descreve como um construto social que derivou em um Programa de Investigação Científica (PIC):

\begin{citacao} 
Nos anos recentes houve uma expansão da atividade do \emph{live coding}, e a formação de um corpo internacional para suportar \emph{live coders} - TOPLAP. O sítio \url{toplap.org} e sua lista de email é a mais ativa casa para práticas artístcas, e o TOPLAP tem sido listado para tais festivais de música eletrônica como Ultrasound (2004), transmediale 2005 (Berlin) e Sonar 2005 (Barcelona) \cite[p.~3-4]{blackwell_programming_2005}.\footnote{Tradução parcial de: \emph{Recent years have seen further expansion of live coding activity, and the formation of an international body to support live coders- TOPLAP (Ward et al. 2004). The toplap.org site and mailing list is the most active current home for this artistic practice, and TOPLAP have been booked for such electronic music festivals as Ultrasound 2004 (Huddersfield), transmediale 2005 (Berlin) and sonar 2005 (Barcelona).}}
\end{citacao}

Atualmente uma outra organização, \emph{Live Code Research Network}\footnote{Disponível em \url{http://www.livecodenetwork.org/}.}, é referência. Esta última promoveu um evento dos anais analisados como uma nuvem de palavras no \autoref{cap:introducao}. 

É interessante notar que, além dos limites institucionais, liminaridades do \emph{livecoding} expandem. Uma delas tange ao uso de programas usados para performance. Por exemplo: a utilização de linguagens visuais, como \emph{PD} ou \emph{Max/Msp}, ao vivo (\emph{live patching}) são um caso à parte ou devem ser incluídas no objeto de pesquisa? É pertinente chamar de \emph{live coding} uma performance que utiliza o \emph{Live}!\emph{Ableton}? Ou apenas atividades em que se codificam textos? Utilizar dispositivos diferentes do \emph{laptpop}, como corpos dançantes, é \emph{live coding}? Tais perguntas apenas expõem as multiplicidades do \emph{livecoding}

Uma pergunta mais específica irá delinear o método de pesquisa, de um universo de conceitos $\mathcal{U}$, contendo infinitos conceitos $c$, para os espaços conceituais definidos na \autoref{eq:universo_pesquisa}. Mais especificamente,\textbf{O que pode ser pressuposto musicalmente e tecnicamente, entre Andrew Sorensen, e seu público, em uma improvisação nomeada \emph{Study in Keith}?}

\begin{equation}
\mathcal{U} = [..., livecoding, \emph{Study in Keith}...]
\end{equation}\label{eq:universo_pesquisa}

\section{Método de pesquisa}\label{conjunto_conhecimentos}

Ao surgir como programa de pesquisa em meados dos anos 2000, \emph{live coders} tendem a praticar uma forma de construção do conhecimento emprestando assuntos de outras áreas. Notamos as seguintes referências aos assuntos: \begin{inparaenum}[(1)]
\item Música;
\item Cinema;
\item Instalações Artísticas;
\item Educação;
\item Ciências da Computação;
\item Semiologia
\item Lógica;
\item Etnografia.
\end{inparaenum}\label{par:metodo1}

Para discutir esta multiplicidade de assuntos, sugerimos o \emph{live coding} como uma feira de idéias, como proposto nos estudos sociais de Boaventura de Souza \citeonline{santos_abissal_2007,santos_filosofia_2008}. Entendemos que pesquisador será forçado, na academia, a limitar aquilo que cada vez mais se expande. Santos discute isso da seguinte maneira:

\begin{citacao}
O saber só existe como pluralidade de saberes, tal como a ignorância só existe como pluralidade de ignorâncias. As possibilidades e os limites de compreensão e de acção de cada saber só podem ser conhecidas na medida em que cada saber se propuser uma comparação com outros saberes. Essa comparação é sempre uma versão contraída da diversidade epistemológica do mundo, já que esta é infinita. É, pois, uma comparação limitada, mas é também o modo de pressionar ao extremo os limites e, de algum modo, de os ultrapassar ou deslocar. Nessa comparação consiste o que designo por ecologia de saberes. (\ldots) Sendo sempre limitado o conjunto de saberes que integra a ecologia dos saberes há que definir como se consitituem esses conjuntos. \textbf{À partida, é possível um número ilimitado de ecologia de saberes, tão ilimitado quanto o da diversidade epistemológica do mundo. Cada exercício de ecologia de saberes implica uma selecção de saberes e um campo de interacção onde o exercício tem lugar}. \cite[p.~28-30]{santos_filosofia_2008}.
\end{citacao}


Para diferenciar a aplicação dos estudos sociais, defino, este trabalho, a \emph{ecologia de saberes} como \textbf{um estudo de um subconjunto de elementos categóricos no universo de conceitos $U$, e suas respectivas derivações}, como demonstrado na \autoref{eq:espaco_conceitual_pesquisa}.

\begin{equation}
\mathcal{C} = <<<T, R, E>>> = [livecoding, \emph{Study in Keith}]
\end{equation}\label{eq:espaco_conceitual_pesquisa}

Se derivarmos o conceito \emph{livecoding} utilizando as \autoref{fig:nuvemlivecoding} e \autoref{tab:comparacao}, teremos uma espaço conceitual multidimensional, ou um outro $\mathcal{U}$ (\autoref{eq:universo_lc}).

\begin{equation}
livecoding = \mathcal{U}_\emph{livecoding} = [\ldots, Collins, Blackwell, McLean, Grossi, \ldots, video, application, support, composition, piece, knowledge, feature, cell, activity, art, \ldots]
\end{equation}\label{eq:universo_lc}


Um desses elementos, \emph{Grossi} destaca a importacia de um estudo dos precedentes históricos. Estes permitem observar o \emph{núcleo de pesquisa} e a \emph{heurística} \cite{lakatos_falsification_1970,neto_lakatos_2008} do \emph{live coding}, e separá-lo, por exemplo, da \emph{live Computer Music} (\autoref{eq:lcm}). Serão estudados no próximo capítulo.

\begin{equation}
livecoding = \mathcal{U}_\emph{computer music performática} = [\ldots, Mathews, GROOVE, Grossi, \ldots]
\end{equation}\label{eq:universo_cmp}

\begin{equation}
livecoding = \mathcal{U}_\emph{live computer music} = [\ldots, Baía de São Franscisco, \ldots]
\end{equation}\label{eq:lcm}

\begin{equation}
livecoding = \mathcal{U}_\emph{toplap} = [\ldots, Collins, Blackwell, McLean, \ldots]
\end{equation}\label{eq:lcm2}

Por sua vez, o espaço conceitual do \emph{live coding} pode ser compreendido como uma união entre os universos de alguns casos paradigmáticos da \emph{Computer Music}, da \emph{live computer music} e do contexto inerente ao universo do TOPLAP.

\begin{equation}
\mathcal{c}_\emph{livecoding} = [\mathcal{U}_\emph{computer music performática}, \mathcal{U}_\emph{live computer music},\mathcal{U}_\emph{toplap} ]
\end{equation}\label{eq:lcm2}

Um paradigma comum entre os elementos da \autoref{eq:lcm2}, é a utilização de um ou mais computador(es), ou partes deles (microchips), para contextos de performance artística.  Neste sentido, podemos separar mais um pouco nosso espaço conceitual (\autoref{eq:espaco_pesquisa2})

\begin{equation}
\mathcal{C}_\emph{pesquisa} = [\mathcal{C}_\emph{livecoding}, Study in Keith]
\end{equation}\label{eq:espaco_conceitual_pesquisa2}

Por sua vez, \emph{Study in Keith} é um outro espaço conceitual que inclui o pianista e compositor Keith Jarret e seus concertos \emph{Sun Bear} (1976-1979).

\begin{equation}
\mathcal{C}_\emph{Study in Keith} = [Keith Jarret, Sun Bear, jazz, improvisação]
\end{equation}\label{eq:espaco_conceitual_pesquisa2}




