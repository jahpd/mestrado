\section{Pergunta para o Método}\label{sec:perguntametodo}

Na Figura 1 apresentei um caso brasileiro. Por um número de documentos reduzidos em língua portuguesa, irei chamar de \emph{live coding} o que é produzido utilizando moajoritariamente nesta língua.

\citeonline{mclean_visualisation_2010} corrobora com definição de  \citeonline{blackwell_programming_2005}. Enquanto primeiro afirma que o \emph{live coding} é uma  "improvisação de vídeo e/ou música usando \textbf{linguagens de computador} que tem se desenvolvido em um campo ativo de pequisa e prática artística ao longo da última década"  \footnote{Tradução parcial nossa de: \emph{Live coding , the improvisation of video and/or music using computer language, has developed into an active field of research and arts pratice over the last decade}. Grifo nosso}, o segundo descreve como um processo social:

\begin{citacao} 
Nos anos recentes houve uma expansão da atividade do \emph{live coding}, e a formação de um corpo internacional para suportar \emph{live coders} - TOPLAP. O sítio \url{toplap.org} e sua lista de email é a mais ativa casa para práticas artístcas, e o TOPLAP tem sido listado para tais festivais de música eletrônica como Ultrasound (2004), transmediale 2005 (Berlin) e Sonar 2005 (Barcelona) \cite[p.~3-4]{blackwell_programming_2005}.\footnote{Tradução parcial de: \emph{Recent years have seen further expansion of live coding activity, and the formation of an international body to support live coders- TOPLAP (Ward et al. 2004). The toplap.org site and mailing list is the most active current home for this artistic practice, and TOPLAP have been booked for such electronic music festivals as Ultrasound 2004 (Huddersfield), transmediale 2005 (Berlin) and sonar 2005 (Barcelona).}}
\end{citacao}

Atualmente um outro sítio, \emph{Live Code Research Network}\footnote{Disponível em \url{http://www.livecodenetwork.org/}.}, é outra referência para o programa de pesquisa. Conforme alternativas organizacionais, técnicas e estéticas do \emph{live coding} proliferam, a tarefa de definir o que é ou não é \emph{live coding} se torna bastante nebulosa. Por exemplo: a utilização de linguagens visuais (\emph{live patching}) são um caso a parte ou devem ser incluídas no objeto de pesquisa? É pertinente chamar de \emph{live coding} uma performance que utiliza o \emph{Live}!\emph{Ableton}? Utilizar dispositivos diferentes do \emph{laptpop}, como corpos dançantes, é \emph{live coding}? 

Sugiro um percursso reformulando a pergunta feita na introdução deste trabalho: \textbf{considerando o \emph{live coding} como uma prática interdisciplinar, o que pode ser pressuposto musicalmente entre os praticantes e o público?} 

\section{Método de pesquisa}\label{conjunto_conhecimentos}

Ao surgir como programa de pesquisa em meados dos anos 2000, \emph{live coders} tendem a praticar uma forma de construção do conhecimento emprestando assuntos de outras áreas. Notei as seguintes referências aos assuntos: \begin{inparaenum}[(1)]
\item Música;
\item Cinema;
\item Instalações Artísticas;
\item Educação;
\item Ciências da Computação;
\item Semiologia
\item Lógica;
\item Etnografia.
\end{inparaenum}\label{par:metodo1}

Representar esta multiplicidade de assuntos, e todas suas assimetrias epistemológicas seria uma tarefa cirúrgica. Empresto o termo do sociólogo português Boaventura de Souza \citeonline{santos_abissal_2007,santos_filosofia_2008}, sugerindo o \emph{live coding} como uma feira de idéias, onde se compra e vende conceitos conforme a necessidade, nesse caso, a própria tese de mestrado.

Reconheço aqui um ponto de vista que observa o objeto de pesquisa como algo assimétrico. Para evitar um corte demasiadamente cirúrgico, Santos sugere um método consciente de pluralidades e desproporções daquilo que se quer saber. O pesquisador será forçado na academia, então, a criar uma abordagem limitada do que se quer falar:

\begin{citacao}
O saber só existe como pluralidade de saberes, tal como a ignorância só existe como pluralidade de ignorâncias. As possibilidades e os limites de compreensão e de acção de cada saber só podem ser conhecidas na medida em que cada saber se propuser uma comparação com outros saberes. Essa comparação é sempre uma versão contraída da diversidade epistemológica do mundo, já que esta é infinita. É, pois, uma comparação limitada, mas é também o modo de pressionar ao extremo os limites e, de algum modo, de os ultrapassar ou deslocar. Nessa comparação consiste o que designo por ecologia de saberes. (\ldots) Sendo sempre limitado o conjunto de saberes que integra a ecologia dos saberes há que definir como se consitituem esses conjuntos. \textbf{À partida, é possível um número ilimitado de ecologia de saberes, tão ilimitado quanto o da diversidade epistemológica do mundo. Cada exercício de ecologia de saberes implica uma selecção de saberes e um campo de interacção onde o exercício tem lugar}. \cite[p.~28-30]{santos_filosofia_2008}.
\end{citacao}


Para diferenciar o termo \emph{ecologia} de sua aplicação nas ciências ecológicas, adaptei para este trabalho como \textbf{estudo do conjunto limitado e desproprocional de categorizações musicais do \emph{live coding}}.


\subsection*{Tarefas}

Tarefas sugeridas para realizar no \autoref{cap:trabalhos_relacionados}: (1) Investigar de onde surgem o núcleo e a heurística do \emph{live coding}  (2) Exposição de uma ideologia musical que mistura:\begin{inparaenum}[(a)]
\item \emph{Música Algorítmica} (que será abreviado como MA)
\item \emph{Música Processual} (MP)
\item \emph{Música Generativa} (MG)
\item \emph{Música de Pista} ou práticas DJ (DJ).
\end{inparaenum}. 

Tarefas sugeridas para realizar no \autoref{cap:resultados}: (1) Contextualização do gênero musical (4) Exposição dos dados.

