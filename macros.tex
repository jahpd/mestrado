% --- 
% CONFIGURAÇÕES DE PACOTES
% --- 

% ---
% Configurações do pacote backref
% Usado sem a opção hyperpageref de backref
\renewcommand{\backrefpagesname}{Citado na(s) página(s):~}
% Texto padrão antes do número das páginas
\renewcommand{\backref}{}
% Define os textos da citação
\renewcommand*{\backrefalt}[4]{
	\ifcase #1 %
		Nenhuma citação no texto.%
	\or
		Citado na página #2.%
	\else
		Citado #1 vezes nas páginas #2.%
	\fi}%
% ---

\newcommand{\traducao}[2]{``#1''\footnote{Tradução nossa de ``\emph{#2}''}}
\newcommand{\tabletraducao}[2]{``#1'' \tablefootnote{Tradução nossa de \emph{#2}.}}
\newcommand{\traducaoparcial}[2]{``#1'' \footnote{Tradução parcial nossa de \emph{#2}.}}
\newcommand{\csf}[2]{$\mathcal{#1}_\emph{#2}$}
\newcommand{\traduzcitacao}[4]{\begin{citacao}
#1\cite[#3]{#4}\footnote{Tradução de \emph{#2}}.
\end{citacao}
}

\newcommand{\disponivelem}[1]{\footnote{Disponível em \url{#1}}}
\newcommand{\sorensen}[2]{#1\disponivelem{#2}}
\newcommand{\ver}[1]{(ver \autoref{#1}, p.~\pageref{#1})}
\newcommand{\traducaoapud}[5]{``#1'' \apud[#3]{#4}{#5}\traducao{#2}}
\newcommand{\pressingtwo}[2]{$\mathcal{#1}'_{#2}$}
\newcommand{\pressingthree}[3]{$\mathcal{#1}'^{#3}_{#2}$} 
\newcommand{\met}[1]{\scriptsize \MakeTextUppercase{#1} \normalsize}
\newcommand{\metafora}[2]{\met{#1}, isto é, #2}
\newcommand{\verex}[1]{(ver Exemplo \autoref{#1}, p.~\pageref{#1})}
\newcommand{\idemibdem}{\emph{idem}, \emph{ibdem}}
\newcommand{\tempo}[2]{#1$'$#2$''$}
\newcommand{\exref}[1]{(ver exemplo \ref{#1})}

%%%%%%%%%%%%%%%%%%%%%%%%%%%%%%%%%%%%%%%%%%%%%%%%
% An example environment
%%%%%%%%%%%%%%%%%%%%%%%%%%%%%%%%%%%%%%%%%%%%%%%%
\theoremheaderfont{\normalfont\bfseries}
\theorembodyfont{\normalfont}
\theoremstyle{break}
\def\theoremframecommand{{\color{deepred}\vrule width 5pt \hspace{5pt}}}
\newshadedtheorem{exa}{Exemplo}[chapter]
\newenvironment{example}[1]{%
		\begin{exa}[#1]
}{%
		\end{exa}
}
