% ---
% PACOTES
% ---
\usepackage{lmodern}			% Usa a fonte Latin Modern
\usepackage[T1]{fontenc}		% Selecao de codigos de fonte.
\usepackage[utf8]{inputenc}		% Codificacao do documento (conversão automática dos acentos)
\usepackage{indentfirst}		% Indenta o primeiro parágrafo de cada seção.
\usepackage{nomencl} 			% Lista de simbolos
\usepackage{color}		               	% Controle das cores
\usepackage{fancyvrb}
\usepackage{graphicx}			% Inclusão de gráficos
\usepackage{txfonts}                	% Fontes virtuais 
\usepackage{listings} 
\usepackage{minted}


% ---
% ABNT
% ---
\usepackage[brazilian,hyperpageref]{backref}    % Paginas com as citações na bibl
\usepackage[alf]{abntex2cite}                  	% Citações padrão ABNT
\usepackage[brazil]{babel}	               	% Idioma do documento


%TODO
\usepackage[colorinlistoftodos]{todonotes}

\makeatletter
\hypersetup{
  pdftitle={\@title},
  pdfauthor={\@author},
  pdfsubject={Seminário para curso de metodologia},
  pdfkeywords={Música}{Metodologia}{Paradigma},
  %pdfcreator={\@LaTeX with \@abnTeX},
  colorlinks=true,
  linkcolor=blue,
  citecolor=blue, 
  urlcolor=blue}
\makeatother

% --- 
% CONFIGURAÇÕES DE PACOTES
% --- 

% ---
% Configurações do pacote backref
% Usado sem a opção hyperpageref de backref
\renewcommand{\backrefpagesname}{Citado na(s) página(s):~}
% Texto padrão antes do número das páginas
\renewcommand{\backref}{}
% Define os textos da citação
\renewcommand*{\backrefalt}[4]{
	\ifcase #1 %
		Nenhuma citação no texto.%
	\or
		Citado na página #2.%
	\else
		Citado #1 vezes nas páginas #2.%
	\fi}%
% ---

\usepackage{tikz}
\tikzset{
  treenode/.style = {shape=rectangle, rounded corners,
                     draw, align=center,
                     top color=white, bottom color=blue!20},
  root/.style     = {treenode, font=\Large, bottom color=red!30},
  env/.style      = {treenode, font=\ttfamily\normalsize},
  dummy/.style    = {circle,draw}
}
