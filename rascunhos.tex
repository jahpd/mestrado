\section{Idéias}

Os escritos dessa seção são idéias espalhadas sobre composição, com uma dose de questionamentos filosóficos.

No presente momento, tenho pensado o \emph{live coding} sobre três aspectos, derivados do ensaio ``Jamais Fomos Modernos'' de \citeonline[p.~64]{latour_jamais_2011}: natureza(técnicas de improvisação através de linguagens de programação); discurso (gêneros musicais); sociedade (mapeamento geográfico através do soundcloud). Algumas notas a respeito do texto ``Terminal Prestige: The case of Avant-garde Music composition 

A própria atitude de ter separado o objeto de pesquisa em três aspectos, pode criar um titã, sob o nome de``super-hiperincomensurabilidade'' do conhecimento. No entanto, este é, segundo o filósofo, um desespero próprio dos pós-modernos\footnote{``Sente que há algo de errado com a crítica, mas não sabe fazer nada além de prolongar a crítica sem no entanto acreditar em seus fundamentos (Lyotard, 1979)'' (p.50)}.

Pergunta crucial: estaria eu agindo como pós-moderno estudando um fenômeno sócio-cultural? Ou seria alguém que nunca foi moderno estudando uma prática com pretensões pós-modernas? Como poderia agir com tal pretensão de abordar os três recursos da crítica (natureza, linguagem, sociedade) se não 